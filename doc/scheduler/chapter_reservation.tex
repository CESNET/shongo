\chapter{Reservation}

Reservations are used for booking devices for videoconference calls between multiple participants. Reservation request can be created by a Shongo user via console or web interface of a domain controller. The user become the reservation owner. Each reservation belongs to a domain controller where it was created and the reservation is stored there. Every domain controller is running a scheduler which process created reservations and tries to schedule them. Through the console or web interface the reservation owner is able to watch the reservation state and also to modify or delete the reservation. The owner and all the reservation participants should be notified about changes in reservation (e.g., by email or SMS). 

Reservation request consists of date/time and duration specifications, specification of requested devices and/or participants that should take part in the videoconference and a set of options that can modify the scheduling process. Reservation request may be used to book multiple videoconference calls (e.g., every Wednesday or two specific dates). Reservation can request following types of entities to take part in the videoconference(s):
\begin{compactitem}
\item Person known to Shongo by a user identity.
\item Person unknown to Shongo.
\item Resource known to Shongo by a resource identifier.
\item Resource unknown to Shongo.
\end{compactitem}

When a person is requested to participate in a videoconference, the controller should contact him (e.g., by email or by SMS with link to a web page) and ask whether he want to accept the invitation. Known persons are specified by a user identity (e.g., eduID.cz). Unknown persons can be anonymous or can be specified by a name and contact (e.g., email or cellphone number). A person specification can also specify a device that the person use to connect the videoconference(s). If a device is not specified the controller should also ask the person which device he will use to connect to the videoconference. If the person is known to Shongo the controller can build a list of available devices and the person can select from them. The person can also specify a device that is unknown to Shongo (at least technology). If the person is unknown to Shongo the controller does not provide a list of available devices but the person should describe the device by which he will connect (again at least technology).  Requested person has always assigned a resource by which the person will connect to the videoconference (even unknown resource with only the technology specified).
\TODO{How to specify different devices to different events (in case of multiple videoconferences)?}

A specific resource can be requested as a part of a reservation request. A resource that is known to Shongo is specified by it's identifier. A resource that is unknown to Shongo is specified at least with it's technology type, but also can be specified a URI, alias or every other attribute that can be specified to known resource in Shongo resource database. If the unknown resource has specified only technology it is anonymous resource (e.g., guest in videoconference).

Set of options that can modify the scheduling process can contain one or more options from the following types:
\begin{compactitem}
\item inter-domain lookup option tells that the controller is allowed to look for necessary devices also in foreign domains (e.g., for a MCU that is not available in local domain).
\item preference whether a terminal should dial a MCU or a MCU should dial a terminal. This preference can be specified globally for the whole reservation or separately for each device.
\end{compactitem}

\TODO{Continue in description of reservations}

\section{Examples of reservation requests}

\CodeStyle{
  PersonSpecification, ResourceSpecification, VirtualRoomSpecification,
}
\CodeStyleAppendImplementation
\CodeStyleAppendEnum

\begin{enumerate}
\item Reservation of a single videoconference for 4 anonymous persons in H.323.

\begin{EntityExample}{Reservation}{reservation1}{}
dateTime: AbsoluteDateTime(2012-05-18T15:30), duration: Period(P1H),
specifications: [
  PersonSpecification { // Four guests (unknown H.323 devices)
    device: {technology: H323},
    count: 4
  }
]
\end{EntityExample}

\item Reservation of a single videoconference for 4 persons. The first person  has assigned a device in resource database and thus the reservation owner select it for the reservation. Other two persons don't have assigned device in resource database so the reservation owner specify to them the H.323 phone numbers that the persons owns and must use (and the owner knows it). One H.323 guest can also connect to the videoconference.

\begin{EntityExample}{DeviceResource}{terminal1}{Mirial for Martin Srom in resource database}
technologies: [H323], 
identity: UserIdentity(srom@cesnet.cz),
capabilities: [
  ReceiveCapability, SendCapability, StandaloneTerminalCapability,
  SignalingClientCapability {aliases: [95001]}
]
\end{EntityExample}

\begin{EntityExample}{Reservation}{reservation2}{}
dateTime: AbsoluteDateTime(2012-05-18T15:30), duration: Period(P1H),
specifications: [
  PersonSpecification { // Martin Srom (known H.323 device)
    identity: UserIdentity(srom@cesnet.cz), 
    device: terminal1
  },
  PersonSpecification { // Petr Holub (unknown device by H.323 phone number)
    identity: UserIdentity(hopet@cesnet.cz), 
    device: {technology: H323, alias: 95002}
  },
  PersonSpecification { // Jan Ruzicka (unknown device by H.323 phone number)
    identity: UserIdentity(janru@cesnet.cz), 
    device: {technology: H323, alias: 95003}
  },
  PersonSpecification { // Guest (unknown H.323 device)
    device: {technology: H323}
  }
]
\end{EntityExample}

\item Reservation of a single videoconference for 3 persons. The persons will be asked which devices they will use to connect (e.g., by email with a link to a web page).

\begin{EntityExample}{Reservation}{reservation3}{}
dateTime: AbsoluteDateTime(2012-05-18T15:30), duration: Period(P1H),
specifications: [
  PersonSpecification { // Martin Srom (must choose a device)
    identity: UserIdentity(srom@cesnet.cz)
  },
  PersonSpecification { // Petr Holub (must choose a device)
    identity: UserIdentity(hopet@cesnet.cz)
  },
  PersonSpecification { // Jan Ruzicka (must choose a device)
    identity: UserIdentity(janru@cesnet.cz)
  }
]
\end{EntityExample}

\item Reservation of a single videoconference that requests virtual room on a MCU for 4 participants. The reservation also requests 2 specific terminals that will connect to the MCU. The remaining two "seats" in the virtual room can be used by any guests.

\begin{EntityExample}{DeviceResource}{mcu1}{H.323 MCU in resource database}
...
\end{EntityExample}

\begin{EntityExample}{DeviceResource}{terminal2}{H.323 terminal in resource database}
...
\end{EntityExample}

\begin{EntityExample}{DeviceResource}{terminal3}{Another H.323 terminal in resource database}
...
\end{EntityExample}

\begin{EntityExample}{Reservation}{reservation4}{}
dateTime: AbsoluteDateTime(2012-05-18T15:30), duration: Period(P1H),
specifications: [
  VirtualRoomSpecification {
    device: mcu1,
    size: 4
  },
  ResourceSpecification {
    resource: terminal2
  },
  ResourceSpecification {
    resource: terminal3
  }
]
\end{EntityExample}

\item Reservation requesting a single videoconference of 10 anonymous persons in H.323. Two MCU devices are specified to be used in videoconference.

\TODO{What does it mean? Both MCU must be used? Or at least one? Or it is only preference and even other MCU may be used?}

\begin{EntityExample}{DeviceResource}{mcu2}{H.323 MCU in resource database}
...
\end{EntityExample}

\begin{EntityExample}{Reservation}{reservation5}{}
dateTime: AbsoluteDateTime(2012-05-18T15:30), duration: Period(P1H),
specifications: [
  ResourceSpecification {
    resource: mcu1
  },
  ResourceSpecification {
    resource: mcu2
  },
  PersonSpecification { // Ten guests (unknown H.323 devices)
    device: {technology: H323},
    count: 10
  }
]
\end{EntityExample}

\end{enumerate}

\TODO{Add more examples of reservation requests}

\section{Implementation}

\CodeStyle{}
\CodeStyleAppendLanguage

\begin{lstlisting}
/**
 * Date/Time objects as defined in API document.
 */
DateTime = Object {
}
AbsoluteDateTime = Object extends DateTime {
}
PeriodicDateTime = Object extends DateTime {
}
Duration = Object {
}

/** 
 * Represents a user identity.
 */
UserIdentity = Object {
}

TODO: Move some classes from scheduler here and complete this
\end{lstlisting}