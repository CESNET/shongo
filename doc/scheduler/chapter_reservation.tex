\chapter{Reservation}

Reservations are used for booking devices for videoconference calls between multiple participants. Reservation request can be created by a Shongo user via console or web interface of a domain controller. The user become the reservation owner. Each reservation belongs to a domain controller where it was created and the reservation is stored there. Every domain controller is running a scheduler which process created reservations and tries to schedule them. Through the console or web interface the reservation owner is able to watch the reservation state and also to modify or delete the reservation. The owner and all the reservation participants should be notified about changes in reservation (e.g., by email or SMS). 

Reservation request consists of date/time and duration specifications, specification of requested devices and/or participants that should take part in the videoconference and a set of options that can modify the scheduling process. Reservation request may be used to book multiple videoconference calls (e.g., every Wednesday or two specific dates). Reservation can request following types of entities to take part in the videoconference(s):
\begin{compactenum}
\item \textbf{Resource known to Shongo} \\
  Resource is stored in the controller database and it has assigned 
  the resource identifier by which is the resource requested in 
  the reservation. Known resources can represent two types of resources:
  \begin{compactenum}
  \item Specific device in the controlled domain. The device is usually 
  managed and thus the controller can control and schedule it.
  \item External device that is often used in reservations. The device is 
  usually external unmanaged terminal device stored in the controller database
  to not force the user to describe it every time when he want to use it in 
  a reservation (actually it is stored version of resource unknown to Shongo).
  \end{compactenum}
\item \textbf{Resource unknown to Shongo} \\
  Resource is described as part of the reservation request. Usually 
  represents an external device. It must specify at least technology (or a set 
  of technologies). For instance it is useful for requesting videoconference 
  of 5 unknown users that can connect via SIP or H.323.
  
  To each requested known/unknown resource in a reservation can be assigned a 
  list of persons, that will use the resource to participate in the 
  videoconference. Known resources can have the list assigned permanently 
  and in this case it is automatically filled into all reservations where the 
  resource is used.  
\item \textbf{Person} \\
  Instead of specifying a resource the reservation can request a specific 
  person to participate in the videoconference and the person himself choose
  the device that he will use for the videoconference. The person can be 
  requested in two ways:
  \begin{compactenum}
  \item By specifying an user identity in Shongo.
  \item By specifying a name and contact (e.g., email or cell phone).
  \end{compactenum}
\end{compactenum}
When a person is requested to participate in a videoconference (by a resource or directly), the controller should contact him (e.g., by email or by SMS with link to a web page) and ask whether he want to accept the invitation. If the person was requested directly the controller should also ask the person which device he will use to connect to the videoconference. If the person was specified by a user identity the controller can build a list of available devices to him and the person can select from them. The person can also specify a device that is unknown to Shongo. 

The videoconference can be scheduled only when all directly requested persons have selected devices by which they will connect. Until this condition is met the reservation is in creating state.
\TODO{How to specify different devices to different events (in case of multiple videoconferences)? Perhaps user should choose a device separately for each event...}

Set of options that can modify the scheduling process of a reservation can contain one or more options from the following types:
\begin{compactitem}
\item inter-domain lookup option tells that the controller is allowed to look for necessary devices also in foreign domains (e.g., for a MCU that is not available in local domain).
\item preference whether a terminal should dial a MCU or a MCU should dial a terminal. This preference can be specified globally for the whole reservation or separately for each device.
\end{compactitem}
Each device resource should specify whether it is callable from other devices or if it should initiate all calls by itself.

\section{Examples of reservation requests}

\CodeStyle{
  ResourceSpecification, PersonSpecification, TerminalSpecification, 
  VirtualRoomSpecification,
}
\CodeStyleAppendImplementation
\CodeStyleAppendEnum

\begin{enumerate}
\item Reservation of a single videoconference for 4 anonymous persons in H.323.

\begin{EntityExample}{Reservation}{reservation1}{}
dateTime: AbsoluteDateTime(2012-05-18T15:30), duration: Period(P1H),
requestedResources: [
  TerminalSpecification { // 4 guests (unknown H.323 devices)
    technology: H323,
    count: 4
  }
]
\end{EntityExample}

\item Reservation of a single videoconference for 4 persons. The first person  has assigned a device in resource database and thus the reservation owner selected it for the reservation. Other two persons don't have assigned device in resource database so the reservation owner specified that they will connect by specific H.323 phone numbers. One H.323 guest can also connect to the videoconference.

\begin{EntityExample}{DeviceResource}{terminal1}{Mirial for Martin Srom in resource database}
technologies: [H323], 
aliases: [{type: E164, value: 95001}],
persons: [UserIdentity(srom@cesnet.cz)],
capabilities: [
  ReceiveCapability, SendCapability, StandaloneTerminalCapability
]
\end{EntityExample}

\begin{EntityExample}{Reservation}{reservation2}{}
dateTime: AbsoluteDateTime(2012-05-18T15:30), duration: Period(P1H),
requestedResources: [
  ResourceSpecification { // Martin Srom (known H.323 device)
    resource: terminal1,
    persons: [UserIdentity(srom@cesnet.cz)] 
  },
  TerminalSpecification { // Petr Holub (unknown H.323 device)
    technology: H323,
    alias: {type: E164, 95002},
    persons: [UserIdentity(hopet@cesnet.cz)]
  },
  TerminalSpecification { // Jan Ruzicka (unknown H.323 device)
    technology: H323,
    alias: {type: E164, 95003},
    persons: [UserIdentity(janru@cesnet.cz)], 
  },
  TerminalSpecification { // Guest (unknown H.323 device)
    technology: H323
  }
]
\end{EntityExample}

\item Reservation of a single videoconference for 3 persons. The persons will be asked which devices they will use to connect (e.g., by email with a link to a web page).

\begin{EntityExample}{Reservation}{reservation3}{}
dateTime: AbsoluteDateTime(2012-05-18T15:30), duration: Period(P1H),
requestedPersons: [
  Person { // Martin Srom (must choose a device)
    userIdentity: UserIdentity(srom@cesnet.cz)
  },
  Person { // Petr Holub (must choose a device)
    userIdentity: UserIdentity(hopet@cesnet.cz)
  },
  Person { // Jan Ruzicka (must choose a device)
    name: Jan Ruzicka,
    email: janru@cesnet.cz
  }
]
\end{EntityExample}

\item Reservation of a single videoconference that requests virtual room on a MCU for 4 participants. The reservation also requests 2 specific terminals that will connect to the MCU. The remaining two "seats" in the virtual room can be used by any guests.

\begin{EntityExample}{DeviceResource}{mcu1}{H.323 MCU in resource database}
...
\end{EntityExample}

\begin{EntityExample}{DeviceResource}{terminal2}{H.323 terminal in resource database}
...
\end{EntityExample}

\begin{EntityExample}{DeviceResource}{terminal3}{Another H.323 terminal in resource database}
...
\end{EntityExample}

\begin{EntityExample}{Reservation}{reservation4}{}
dateTime: AbsoluteDateTime(2012-05-18T15:30), duration: Period(P1H),
requestedResources: [
  VirtualRoomSpecification {
    device: mcu1,
    size: 4
  },
  ResourceSpecification {
    resource: terminal2
  },
  ResourceSpecification {
    resource: terminal3
  }
]
\end{EntityExample}

\item Reservation requesting a single videoconference for 1 specific person that should choose a device by which he will connect and 10 anonymous persons in H.323. Two MCU devices are specified to be used in videoconference.

\TODO{What does it mean? Both MCU must be used? Or at least one? Or it is only preference and even other MCU may be used?}

\begin{EntityExample}{DeviceResource}{mcu2}{H.323 MCU in resource database}
...
\end{EntityExample}

\begin{EntityExample}{Reservation}{reservation5}{}
dateTime: AbsoluteDateTime(2012-05-18T15:30), duration: Period(P1H),
requestedPersons: [
  Person { // Martin Srom (must choose a device)
    userIdentity: UserIdentity(srom@cesnet.cz)
  }
],
requestedResources: [
  ResourceSpecification {
    resource: mcu1
  },
  ResourceSpecification {
    resource: mcu2
  },
  TerminalSpecification { // 10 guests (unknown H.323 devices)
    technology: H323,
    count: 10
  }
]
\end{EntityExample}

\end{enumerate}

\TODO{Add more examples of reservation requests}

\section{Implementation}

\CodeStyle{}
\CodeStyleAppendLanguage

\begin{lstlisting}
/**
 * Date/Time objects as defined in API document.
 */
DateTime = Object {
}
AbsoluteDateTime = Object extends DateTime {
}
PeriodicDateTime = Object extends DateTime {
}
Period = Object {
}

/**
 * Represents a time slot in a schedule defined by a start 
 * date/time and and a duration.
 */
TimeSlot = Object {
  // Absolute date/time when the event takes place.
  dateTime: AbsoluteDateTime,

  // Duration which the event takes.
  duration: Duration
}

/** 
 * Represents a user identity.
 */
UserIdentity = Object {
}

/**
 * Represents a person for Shongo.
 */
Person = Object {
  name: String,
  email: String,
  userIdentity: UserIdentity
}

TODO: Move some classes from scheduler here and complete this
\end{lstlisting}