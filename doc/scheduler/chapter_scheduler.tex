\chapter{Scheduler}

Scheduler is a component of a domain controller that has access to a database 
of domain resources and reservations. Scheduler allocates the requested 
resources by reservations to specific time slots and specific resources 
according to requested parameters and optimization criteria.

\section{Scheduling}

Scheduler holds the resource database and waits for incoming reservations. When a reservation arise, the scheduler performs scheduling operation. Each reservation can request multiple reservation events. So the first step that
scheduler should do with a new reservation is to break it (enumerate it) to
all reservation events. If infinite reservations are allowed the scheduling must run with a set maximum future date to which the planning will be done.

\paragraph{Scheduling of a reservation event}
{
\renewcommand{\labelenumi}{\arabic{enumi}.}
\renewcommand{\labelenumii}{\arabic{enumi}.\arabic{enumii}}
\begin{compactenum}
\item ...
  \begin{compactenum}
  \item ...
  \item ...
  \end{compactenum}
\item ...
\end{compactenum}
}

\CodeInput{code_scheduler.txt}

\TODO{Continue in describing scheduler}

\paragraph{Scheduler optimization criteria:}
\begin{compactitem}
\item Minimize resource utilization
\item Maximize available continuous time slots for resources (minimize 
  fragmentation)
\end{compactitem}
