\chapter{Domain Controller}

Domain controller holds the database of resources, reservation/meeting requests 
and allocated resources. When user creates/modifies/deletes a reservation 
request a domain controller should create/modify/delete corresponding meeting 
request(s). Domain controller should ask requested persons from incomplete 
meeting requests for a confirmation and/or device selection. When all persons 
from an incomplete meeting accepts/rejects a meeting request it become the
complete request. Complete requests are processed by a scheduler.

\section{Scheduler}

Scheduler is a component of a domain controller that has access to a database 
of domain resources and complete meeting requests. The scheduler watches for
created/modified/deleted complete meeting requests. When a new request arise
the scheduler tries to schedule it. When the scheduler finds a modified request
it tries to reschedule it (\TODO{vyraz pro preplanovani s minimem zmen}). When 
a deleted request arise, the scheduler free all allocated resources. Scheduling of a meeting request is composed of several steps:
\begin{compactenum}
\item Lookup of partially specified devices.
\item Lookup of local/foreign infrastructure devices to interconnect all requested devices.
\item Allocating available resources.
\item Informing request owner about the state.
\end{compactenum}
The scheduling is driven by optimization criteria and options specified in the request itself.

\TODO{Continue in describing scheduler}

\paragraph{Scheduler optimization criteria:}
\begin{compactitem}
\item Minimize resource utilization
\item Maximize available continuous time slots for resources (minimize 
  fragmentation)
\end{compactitem}

\CodeInput{code_scheduler.txt}
