\documentclass[a4paper]{report}
\usepackage{geometry}
\geometry{paper=a4paper}

\usepackage[T1]{fontenc}
\usepackage[utf8]{inputenx}
\usepackage{palatino}
\usepackage{mathpazo}
\usepackage{microtype}
\renewcommand*\ttdefault{txtt}

\usepackage[czech,english]{babel}

\usepackage[pdftex,breaklinks=true,pdfborder={0 0 0}]{hyperref}
\usepackage[pdftex]{graphicx}
\usepackage{paralist} 
\usepackage{multicol}

\usepackage{tikz}
\usetikzlibrary{snakes,arrows,shapes}
\usepackage{amsmath}

\usepackage{xcolor}
\newcommand{\TODO}[1]{%
\def\empty{}%
\def\prvniparametr{#1}%
\ifx\prvniparametr\empty%
\begingroup\tt\textcolor{red}{\noindent\textbf{TODO}}\endgroup
\else%
\begingroup\tt\textcolor{red}{\noindent\textbf{TODO:}\ #1}\endgroup
\fi%
}

% Include graph (label, filename, name, scale)
\newcommand{\graph}[4]{
\begin{figure}[h!]
\centering\scalebox{\ifx&#4& 0.75 \else #4 \fi}{\input{#2}}
\label{#1}
\caption{#3}
\end{figure}
}


\begin{document}


\title{Scheduler for Shongo}
\author{Petr Holub, Jan Růžička, Miloš Liška, Martin Šrom, Ondřej Pavelka, Ondřej Bouda}
\date{\copyright~CESNET~z.\,s.\,p.\,o.\\2012}
\maketitle
\tableofcontents


\chapter{Topology}

Topology is in-memory representation of videoconferencing devices and it's relations. Each domain controller will keep the local topology of all devices that the controller manages.
Controller will know nothing about topologies from foreign domains. Every time when the controller will need some information about foreign devices it send request to foreign domain controller.

Topology is abstract representation of all technology dependent topologies (e.g, H.323, SIP, Adobe Connect topologies).

\section{Elements}

Type of elements in topology:
\begin{itemize}

\item{\textbf{Node}} 

Each node represents one device.
Each node gets assigned input and output capacity (e.g., the limitation of 
a number of incoming/outgoing connections, bandwidth, etc.) for every channel
that it supports.

Terminal devices will limit the number of input/output connections for each
channel to 1.

%Multipoint devices will have... TODO:
   
% Managed device gets assigned agent, that manages the device. The agent
% is identified by NAME which is part of agent identifier in JADE 
% middle-ware (NAME@DOMAIN).  

\item{\textbf{Oriented Edge}} 

Edge links two nodes (devices) and it represents the 
ability to transmit data (or to make a call) between these two devices 
in specified direction.
  
Each edge can have the following capabilities:
\begin{compactitem}   
  \item \textbf{Audio and Video Channel} -- Specify whether an audio and video channel is available between the two devices.
      
  \item \textbf{Content Channel} -- Specifies whether a content channel is available between two devices.
    
  \item \textbf{Weight} -- Composed of several components:
  \begin{compactitem}  
    \item Inter-domain/Intra-domain
    \item Distance (roughly)
  \end{compactitem}  
\end{compactitem}   

\end{itemize}

\TODO{Topology should keep the information about already scheduled reservations.}
  

\section{Implementation}
Topology implementation should allow dynamic modification when for instance some device parameters are changed to not force the whole topology reconstruction.
      
\subsection{Construction}

Construction is performed when the controller is started. Steps to construct a topology:

\begin{enumerate}

\item Get the list of all devices with it's capabilities in a topology.

\item For each device: 
  \begin{compactitem}
  \item Find all other devices that the device is able to transmit
  data to (or to make a call) and for each:
    \begin{compactitem}
    \item Add edge with proper parameters.
    \end{compactitem}
  \end{compactitem}

\end{enumerate}     

     
\subsection{Update Topology}     

Steps to update a topology when the capabilities for a device are changed:
\begin{enumerate}

\item Remove all edges that reference the device.

\item Find all other devices that the device is able to transmit data from/to
   (or to make a call) and for each:
   \begin{compactitem}
   \item Add edge(s) with proper parameters.
   \end{compactitem}

\end{enumerate}  


\section{H.323 Topology}

The H.323 network is composed of terminals, multipoint control units, gateways and gatekeepers. Terminals, MCUs and gateways are referred to as endpoints. At least two terminals are needed to start a videoconference call.

\begin{itemize}

\item \textbf{Terminal} \\
Terminal represents a hardware device or a software client that enables a single person or a group of persons situated in front of the device to connect to another H.323 endpoint and take a videoconference call.

\item \textbf{Multipoint Control Unit} (MCU) \\
MCU represents a hardware device that is used for hosting of videoconference calls in which can generally participate any number of endpoints. MCU contains one or more \textbf{virtual rooms} and each virtual room can host one videoconference call at the time. MCU can restrict the number of participants per virtual room or per whole MCU device.

\item \textbf{Gateway} \\
Gateway is device that enable communication between H.323 network and other network (e.g., PSTN).

\item \textbf{Gatekeeper} \\
Gatekeeper is a hardware device that provides special services to the other elements in the network: endpoint registration, address resolution, admission control and user authentication.

\end{itemize}

Each H.323 element is running on specific IP address that can be used to connect to the element. An element can have assigned alias, \textbf{H.323 phone number} (sequence of digits) or \textbf{H.323 identifier} (string, e.g., email), that can be used to connect to it. It is useful when the IP address is not known (e.g., it is behind the NAT) or it is likely to change (organization changes).

Alias can be assigned only to \textbf{terminal} or \textbf{virtual room} in MCU. When an terminal has assigned alias it must register to one \textbf{gatekeeper} that will be able to translate the alias to the terminal IP address. When an element want to connect to another terminal or virtual room by alias, it must use the same gatekeeper that has the proper registry of aliases.

When a gatekeeper is used in videoconference, it can work in two modes:
\begin{itemize}
\item Direct Endpoint Model -- The gatekeeper is used only for initialization (address resolution, admission control, etc.) and the videoconference call itself is managed by it's participants (endpoints). The gatekeeper doesn't have full control over the call signaling.
\item Gatekeeper Routed Model -- The gatekeeper is used throughout videoconference call, the call signaling flow through the gatekeeper and it has the full control over it.
\end{itemize}

Each terminal and MCU element can have set zero or one gatekeeper. If gatekeeper is set then each terminal and virtual room in MCU can have an alias set. Each terminal and MCU element should also be configured to allow or forbid the direct connections through IP address. When the gatekeeper with admission control is set, it is appropriate to forbid direct connections through IP address.


\subsection{Examples}

\subsubsection{Connection through direct IP addresses}

Terminals and virtual rooms in MCU don't have assigned any H.323 aliases thus the connecting is done through IP addresses. This topology allows for 2-point videoconferences (e.g., \emph{terminal1 -- terminal2}) and multipoint videoconferences through the MCU (e.g., \emph{terminal1 -- terminal2 -- terminal3 -- mcu}). When an terminal connects to the MCU it must select the proper virtual room (if more is present).

\graph{graph:h323:direct}{graph/h323_direct.tex}{Connection through direct IP addresses}{}


\subsubsection{Connection through gatekeeper}

Terminals and MCUs are configured to forbid direct connections through IP addresses. Terminals and MCU virtual rooms have assigned H.323 aliases and the devices register to the gatekeeper that is able to resolve these aliases to direct IP addresses. When terminal want to start videoconference he must request the gatekeeper.

\graph{graph:h323:gatekeeper}{graph/h323_gatekeeper.tex}{Connection through gatekeeper}{}


\subsubsection{Connection through gatekeeper and direct IP addresses}

These topology is combination of previous two topologies. Terminals and MCUs are configured to allow direct connections through IP addresses. Terminals and MCU virtual rooms have assigned H.323 aliases. Videoconference can be started by direct connection throught IP address or through the gatekeeper.

\graph{graph:h323:gatekeeper}{graph/h323_gatekeeper_or_direct.tex}{Connection through gatekeeper or direct IP addresses}{0.7}


\subsubsection{Hybrid -- Connections through gatekeeper and/or direct IP addresses}

\TODO{Describe} \\
\TODO{Do we allow more gatekeepers in one controller domain?}


\chapter{Scheduler}

Scheduler will use described topology for scheduling reservations of videoconference devices.
\TODO{}


\end{document}