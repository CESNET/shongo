\documentclass[a4paper]{report}
\usepackage{geometry}
\geometry{paper=a4paper}

\usepackage[T1]{fontenc}
\usepackage[utf8]{inputenx}
\usepackage{palatino}
\usepackage{mathpazo}
\usepackage{microtype}
\renewcommand*\ttdefault{txtt}

\usepackage[czech,english]{babel}

\usepackage[pdftex,breaklinks=true,pdfborder={0 0 0}]{hyperref}
\usepackage[pdftex]{graphicx}
\usepackage{paralist} 
\usepackage{multicol}

\usepackage{tikz}
\usetikzlibrary{snakes,arrows,shapes}
\usepackage{amsmath}
\usepackage[numbers]{natbib}

\usepackage{xcolor}
\newcommand{\TODO}[1]{%
\def\empty{}%
\def\prvniparametr{#1}%
\ifx\prvniparametr\empty%
\begingroup\tt\textcolor{red}{\noindent\textbf{TODO}}\endgroup
\else%
\begingroup\tt\textcolor{red}{\noindent\textbf{TODO:}\ #1}\endgroup
\fi%
}

% Include graph (label, filename, name, scale)
\newcommand{\graph}[4]{
\begin{figure}[h!]
\centering\scalebox{\ifx&#4& 0.7 \else #4 \fi}{\input{#2}}
\label{#1}
\caption{#3}
\end{figure}
}


\begin{document}


\title{Scheduler for Shongo}
\author{Petr Holub, Jan Růžička, Miloš Liška, Martin Šrom, Ondřej Pavelka, Ondřej Bouda}
\date{\copyright~CESNET~z.\,s.\,p.\,o.\\2012}
\maketitle
\tableofcontents


\chapter{Technology}

This chapter describes widely-used videoconference technologies. Element types and network examples are listed  for each technology.


\section{H.323}

The H.323 network is composed of terminals, multipoint control units, gateways and gatekeepers. Terminals, MCUs and gateways are referred to as endpoints. At least two terminals are needed to start a videoconference call.

\subsection{Elements}

\begin{itemize}

\item \textbf{Terminal} \\
Terminal represents a hardware device or a software client that enables a single person or a group of persons situated in front of the device to connect to another H.323 endpoint and take a videoconference call.

There are special terminal devices that provides a single virtual room to which can connect multiple other endpoints.  These terminals thus allow to make a multipoint videoconference even without the MCU.

\item \textbf{Multipoint Control Unit} (MCU) \\
MCU represents a hardware device that is used for hosting of videoconference calls in which can generally participate any number of endpoints. MCU contains one or more \textbf{virtual rooms} and each virtual room can host one videoconference call at the time. MCU can restrict the number of participants per virtual room or per whole MCU device.

\item \textbf{Gateway} \\
Gateway is device that enable communication between H.323 network and other network (e.g., PSTN).

\item \textbf{Gatekeeper} \\
Gatekeeper is a hardware device that provides special services to the other elements in the network: endpoint registration, address resolution, admission control and user authentication.

\end{itemize}

Each H.323 element is running on specific IP address that can be used to connect to the element. An element can have assigned alias, \textbf{H.323 phone number} (sequence of digits) or \textbf{H.323 identifier} (string, e.g., email), that can be used to connect to it. It is useful when the IP address is not known (e.g., it is behind the NAT) or it is likely to change (organization changes).

Alias can be assigned only to \textbf{terminal} or \textbf{virtual room} in MCU. When an terminal has assigned alias it must register to one \textbf{gatekeeper} that will be able to translate the alias to the terminal IP address. When an element want to connect to another terminal or virtual room by alias, it must use the same gatekeeper that has the proper registry of aliases.

When a gatekeeper is used in videoconference, it can work in two modes:
\begin{itemize}
\item Direct Endpoint Model -- The gatekeeper is used only for initialization (address resolution, admission control, etc.) and the videoconference call itself is managed by it's participants (endpoints). The gatekeeper doesn't have full control over the call signaling. The target IP address is communicated to the caller.
\item Gatekeeper Routed Model -- The gatekeeper is used throughout videoconference call, the call signaling flow through the gatekeeper and it has the full control over it. The target IP address is not communicated to the caller.
\end{itemize}

Each terminal and MCU element can have set zero or one gatekeeper. If gatekeeper is set then each terminal and virtual room in MCU can have an alias set. Each terminal and MCU element should also be configured to allow or forbid the direct connections through IP address. When the gatekeeper with admission control is set, it is appropriate to forbid direct connections through IP address.

\subsection{Constraints}

\begin{itemize}

\item Communication consists of:
\begin{compactenum}
\item Audio
\item Audio + Video
\item Audio + Video + Data
\end{compactenum}

\item Always bidirectional connections. When an device $A$ can connect to a device $B$ then also $B$ can connect to $A$.
\\ \TODO{How about the case when A can send only audio to B but B can send \\ audio+video+content?}
\\ \TODO{I think we should allow asymmetric links... There may be terminals that for instance don't support data streams (behind the NAT)}

\end{itemize}

\subsection{Network Examples}

\subsubsection{Connection through direct IP addresses}

Terminals and virtual rooms in MCU don't have assigned any H.323 aliases thus the connecting is done through IP addresses. This topology allows for 2-point videoconferences (e.g., \emph{terminal1 -- terminal2}) and multipoint videoconferences through the MCU (e.g., \emph{terminal1 -- terminal2 -- terminal3 -- mcu}). When an terminal connects to the MCU it must select the proper virtual room (if more is present).

\graph{graph:h323:direct}{graph/h323_direct.tex}{Connection through direct IP addresses}{}

\subsubsection{Connection through gatekeeper}

Terminals and MCUs are configured to forbid direct connections through IP addresses. Terminals and MCU virtual rooms have assigned H.323 aliases and the devices register to the gatekeeper that is able to resolve these aliases to direct IP addresses. When terminal want to start videoconference he must request the gatekeeper. The request must contain H.323 alias of another terminal or virtual room in MCU to which the terminal want to connect.

\graph{graph:h323:gatekeeper}{graph/h323_gatekeeper.tex}{Connection through gatekeeper}{}

\subsubsection{Connection through gatekeeper or direct IP addresses}

These topology is combination of previous two topologies. Terminals and MCUs are configured to allow direct connections through IP addresses. Terminals and MCU virtual rooms have assigned H.323 aliases. Videoconference can be started by direct connection throught IP address or through the gatekeeper.

\graph{graph:h323:gatekeeper}{graph/h323_gatekeeper_or_direct.tex}{Connection through gatekeeper or direct IP addresses}{}

\subsubsection{Multiple gatekeepers}

The network is composed from 3 endpoints (\emph{terminal1, terminal2, mcu}) managed by \emph{gatekeeper1} and 2 subdomains where each is composed of 2 endpoints and one gatekeeper. 
\graph{graph:h323:gatekeeperMultiple}{graph/h323_gatekeeper_multiple.tex}{Multiple gatekeepers}{0.6}
The network is composed from three local networks: main domain, subdomain1 and subdomain2. Adjacent endpoints (endpoints in the same local network) can connect directly or through it's gatekeeper. Not adjacent endpoints (endpoints from different local networks) can connect only by it's gatekeepers.
\\ \TODO{Do we allow more gatekeepers in one controller domain?}
\\ \TODO{How to enable direct connections inside subdomains and disable them outside subdomains?}

\subsection{References}

\renewcommand{\bibsection}{}
\begin{thebibliography}{1}

\bibitem[1]{bib:h323:architecture}
Basic Architecture of H.323.
\\ \url{http://hive1.hive.packetizer.com/users/packetizer/papers/h323/h323_basics_handout.pdf}

\bibitem[2]{bib:h323:wiki}
H.323 on Wikipedia.
\\ \url{https://en.wikipedia.org/wiki/H.323}

\end{thebibliography}


\section{SIP}

\TODO{Describe SIP}

\subsection{Elements}

Type of elements in SIP network:

\begin{itemize}
\item \textbf{User Agent} \\
TODO
\item \textbf{Proxy Server} \\
TODO
\item \textbf{Redirect Server} \\
TODO
\item \textbf{Registrar} \\
TODO
\end{itemize}

\subsection{Constraints}

\begin{itemize}

\item Communication consists of:
\begin{compactenum}
\item Audio
\item Audio + Video
\item Audio + Video + Data
\end{compactenum}

\end{itemize}

\subsection{Network Examples}

\subsection{References}

\renewcommand{\bibsection}{}
\begin{thebibliography}{1}

\bibitem[1]{bib:sip:architecture}
Session Initiation Protocol (SIP)
\\ \url{http://www.vide.net/cookbook/cookbook.en//list_page.php?topic=3&url=sip.htm}

\bibitem[2]{bib:sip:wiki}
Session Initiation Protocol on Wikipedia.
\\ \url{https://en.wikipedia.org/wiki/Session_Initiation_Protocol}

\end{thebibliography}


\section{Adobe Connect}

\TODO{Describe Adobe Connect}


\chapter{Abstract Topology}

Abstract topology is representation of videoconferencing devices from various technologies and it's relations. Each domain controller in Shongo will keep in memory the local topology of all devices that the controller manages.
Controller will know nothing about topologies from foreign domains. Every time when the controller needs some information about foreign devices it send request to foreign domain controller.

Abstract topology is composed of nodes and edges. Nodes represents a videoconference devices and and edges theirs relations, ability to communicate.

\section{Node}

Each node in Abstract Topology represents one videoconference device. Device is able to communicate with one or more technologies (H.323, SIP, Adobe Connect, etc.).
\\ 
If device is able to communicate with a technology it means that it has capability to:
\TODO{Finish node capabilities}
\begin{compactitem}
%\item \textbf{send} and/or \textbf{receive} the technology \textbf{data} in \textbf{number} of streams. 
%\textbf{audio} in a specific \textbf{audio codec} and \textbf{bitrate}.

\item \textbf{send} and/or \textbf{receive} the technology \textbf{audio} in a specific \textbf{audio codec} and \textbf{bitrate}.
\item \textbf{send} and/or \textbf{receive} the technology \textbf{video} in a specific \textbf{video codec} and \textbf{resolution}.
\item \textbf{send} and/or \textbf{receive} the technology \textbf{data} in a specific \textbf{data format}.
\item \textbf{mix} received streams (audio, video or data)
\end{compactitem}
If device is able to communicate with a multiple technologies it can also have the following capabilities:
\begin{compactitem}
\item translate the technology \textbf{audio} in a specific \textbf{audio codec} and \textbf{bitrate} to another technology \textbf{audio} in a specific \textbf{audio codec} and \textbf{bitrate}.
\end{compactitem}

Each device can have one or more described capabilities.
   
% Managed device gets assigned agent, that manages the device. The agent
% is identified by NAME which is part of agent identifier in JADE 
% middle-ware (NAME@DOMAIN).  

\section{Edge}

Edge links two nodes (devices) and it represents the 
ability to transmit data (or to make a call) between these two devices 
in specified direction.
  
Each edge can have the following capabilities:

\TODO{Finish edge capabilities}

\begin{compactitem}   
  \item \textbf{Audio and Video Channel} -- Specify whether an audio and video channel is available between the two devices.
      
  \item \textbf{Content Channel} -- Specifies whether a content channel is available between two devices.
    
  \item \textbf{Weight} -- Composed of several components:
  \begin{compactitem}  
    \item Inter-domain/Intra-domain
    \item Distance (roughly)
  \end{compactitem}
    
\end{compactitem}   

\TODO{Topology should keep the information about already scheduled reservations.}
  

\chapter{Technology Implementation in Abstract Topology}
Each videoconference network in a specific technology can be implemented by Abstract Topology. This chapter shows examples of network implementation for each technology.

\TODO{Show examples}

\section{H.323}

\section{SIP}

\section{Adobe Connect}


\chapter{Abstract Topology Implementation}

\TODO{Describe implementation better}

Topology implementation should allow dynamic modification when for instance some device parameters are changed to not force the whole topology reconstruction.      
      
\section{Construction}

Construction is performed when the controller is started. Steps to construct a topology:

\begin{enumerate}

\item Get the list of all devices with it's capabilities in a topology.

\item For each device: 
  \begin{compactitem}
  \item Find all other devices that the device is able to transmit
  data to (or to make a call) and for each:
    \begin{compactitem}
    \item Add edge with proper parameters.
    \end{compactitem}
  \end{compactitem}

\end{enumerate}     
     
\section{Update Topology}     

Steps to update a topology when the capabilities for a device are changed:
\begin{enumerate}

\item Remove all edges that reference the device.

\item Find all other devices that the device is able to transmit data from/to
   (or to make a call) and for each:
   \begin{compactitem}
   \item Add edge(s) with proper parameters.
   \end{compactitem}

\end{enumerate}  


\chapter{Scheduler}

Scheduler will use described abstract topology implementation for scheduling reservations of videoconference devices.

\TODO{Describe scheduler}


\end{document}