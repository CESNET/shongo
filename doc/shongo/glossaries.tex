% Entries
\newglossaryentry{controller}
{
  name=Controller,
  description={Represents a running application that holds a database of 
    resources, user requests and allocated reservations for a single 
    domain. Users can access and modify the database through the controller 
    user interface(s). The controller runs a scheduler that processes the 
    database.}
}
\newglossaryentry{connector}
{
  name=Connector,
  description={Represents a running application that controls a single 
    video/web conferencing hardware or software equipment and provides an API 
    that allows the controller to access that equipment.}
}
\newglossaryentry{scheduler}
{
  name=Scheduler,
  description={Scheduler is a component of a domain controller that processes 
    complete compartment requests and schedule them to allocated compartments 
    which forms reservations.}
}
\newglossaryentry{user}
{
  name=User,
  description={Represents an authenticated person that can access a domain 
    controller through it's user interface(s).}
}
\newglossaryentry{date-time-slot}
{
  name=Date/Time Slot,
  description={Represents a time between specific starting and ending moment. 
    Can be specified by a starting date and time and a duration. In case of
    periodic date/times, the date/time slot term represents an one specific 
    date/time referred to as absolute date/time.}
}
\newglossaryentry{technology}
{
  name=Technology,
  description={Represents a single video/web conferencing technology (e.g., 
    H.323, SIP, or Adobe Connect).}
}
\newglossaryentry{resource}
{
  name=Resource,
  description={An entity that can be requested for a reservation and scheduled 
  by a scheduler.}
}
\newglossaryentry{device}
{
  name=Device,
  description={A special type of resource that represents a video/web 
    conferencing hardware or software equipment (e.g., H.323 terminal, H.323 
    MCU, Adobe Connect server, gateway or streaming server).}
}
\newglossaryentry{reservation-request}
{
  name=Reservation Request,
  description={A request that is made by an user to book resources for one or 
    more videoconference calls. Requested resources and/or persons can be 
    specified in compartments (by default a single compartment is used). 
    From a single reservation request the domain controller creates 
    one or more compartment requests in dependence on the 
    number of specified date/time slots and the number of specified 
    compartments in the reservation request. For each compartment at each 
    date/time slot is created one compartment request.}
}
\newglossaryentry{compartment}
{
  name=Compartment,
  description={Represents a specification of a group of resources and/or 
    persons which should be interconnected by a scheduler. Each reservation 
    request can specify one or more compartments. The compartment is not
    bound to a specific date/time slot. The controller automatically expands 
    each compartment to one or more compartment requests. For each compartment 
    at each absolute date/time slot specified in the reservation request is 
    created one compartment request and thus the compartment request is bound 
    to a specific date/time slot (unlike the compartment).}
}
\newglossaryentry{compartment-request}
{
  name=Compartment Request,
  description={Represents a specification for a group of resources and/or
    persons that should participate in a single videoconference call at 
    specific date/time slot. Compartment request can be \textbf{incomplete} 
    or \textbf{complete}. Incomplete compartment request must be filled in by 
    additional information to become complete (e.g., by asking persons for 
    confirmation or to choose a terminal).}
}
\newglossaryentry{allocated-compartment}
{
  name=Allocated Compartment,
  description={Represents a complete compartment request for which the 
    scheduler has already allocated resources. Allocated compartment 
    is a result from scheduler.}
}
\newglossaryentry{reservation}
{
  name=Reservation,
  description={Represents a list of successfully allocated compartments
    based on a single reservation request.}
}

% Print glossaries
\renewcommand*{\glossaryname}{List of Terms}  % Changes glossary name
\renewcommand*{\glspostdescription}{}         % removes dot after description
\renewcommand*{\glsnamefont}[1]{\textbf{#1}}  % makes name bold
\renewcommand*{\arraystretch}{1.4}            % makes vertical space between entries
                                              % add the table of contents entry
\renewcommand*\glossarypreamble{\addcontentsline{toc}{chapter}{\glossaryname}}
\setlength{\glspagelistwidth}{0.1\linewidth}
\setlength\glsdescwidth{0.7\linewidth}        % sets width of descriptions
\glsaddall                                    % add all glossaries
\glossarystyle{long}                          % sets glossary style to table
\renewcommand*{\glsgroupskip}{}               % remove vertical space between groups
\printglossary[type=main]                     % print glossaries

