% Entries
\newglossaryentry{g:shongo}
{
  name=Shongo,
  description={Represents a distributed multimedia resource management system.}
}
\newglossaryentry{g:domain}
{
  name=domain,
  description={Represents an organization which can run it's own \gls{g:controller} and participate in \gls{g:shongo}.}
}
\newglossaryentry{g:controller}
{
  name=controller,
  description={Represents an application that holds a database of 
    \glspl{g:resource}, \glspl{g:reservation-request} and \glspl{g:reservation} 
    for a single \gls{g:domain}. \Glspl{g:user} can access and modify the database 
    through a \gls{g:controller-client}. The controller runs a \gls{g:scheduler}
    which allocates \glspl{g:reservation-request} to \glspl{g:reservation}.}
}
\newglossaryentry{g:controller-client}
{
  name=controller client,
  description={Represents an application (e.g., command-line or web) which is able to
    connect to a \gls{g:controller} and perform commands through the \gls{g:controller}'s API.}
}
\newglossaryentry{g:resource}
{
  name=resource,
  description={An entity that can be requested for a reservation and allocated 
  by a \gls{g:scheduler}.}
}
\newglossaryentry{g:resource-device}
{
  name=device resource,
  description={A special type of \gls{g:resource} that represents a video/web 
    conferencing hardware or software equipment (e.g., H.323 terminal, H.323 
    MCU, Adobe Connect server, gateway or streaming server).}
}
\newglossaryentry{g:reservation-request}
{
  name=reservation request,
  description={A request that is made by an \gls{g:user} to book \gls{g:resource}(s)
    for a specific \gls{g:date-time-slot}(s). Reservation request can be \emph{incomplete} 
    or \emph{complete}. Incomplete requests must be filled in by 
    additional information to become complete. \Gls{g:scheduler} processes only 
    complete reservation requests.}
}
\newglossaryentry{g:reservation}
{
  name=reservation,
  description={Represents allocated \gls{g:resource}(s) for a complete 
    \gls{g:reservation-request}. \Glspl{g:reservation} are created by a \gls{g:scheduler}.}
}
\newglossaryentry{g:user}
{
  name=user,
  description={Represents an authenticated person that can access a \gls{g:controller} 
    through a \gls{g:controller-client}.}
}
\newglossaryentry{g:scheduler}
{
  name=scheduler,
  description={Scheduler is a component of a \gls{g:controller} which processes 
    complete \glspl{g:reservation-request} and allocates them to \glspl{g:reservation}.}
}
\newglossaryentry{g:connector}
{
  name=connector,
  description={Represents an application that runs one or multiple \glspl{g:connector-agent}.}
}
\newglossaryentry{g:connector-agent}
{
  name=connector agent,
  description={Represents an application that manages a single \gls{g:resource-device} 
    and provides an API that allows the \gls{g:controller} to access that equipment.}
}
\newglossaryentry{g:date-time-slot}
{
  name=date/time slot,
  description={Represents a time between specific starting and ending moment. 
    Can be specified by a starting date and time and a duration. In case of
    periodic date/times, the date/time slot term represents one specific 
    date/time referred to as absolute date/time.}
}
\newglossaryentry{g:technology}
{
  name=technology,
  plural=technologies,
  description={Represents a single video/web conferencing technology (e.g., 
    H.323, SIP, or Adobe Connect).}
}
\newglossaryentry{g:compartment}
{
  name=compartment,
  description={Represents a group of endpoints and/or persons which participate 
    in a single video/web conference but the conference can spread out through multiple 
    virtual rooms and even through multiple \glspl{g:technology} (when gateway device is used).}
}


% Print glossaries
\renewcommand*{\glossaryname}{List of Terms}  % Changes glossary name
\renewcommand*{\glspostdescription}{}         % removes dot after description
\renewcommand*{\glsnamefont}[1]{\textbf{#1}}  % makes name bold
\renewcommand*{\arraystretch}{1.4}            % makes vertical space between entries
                                              % add the table of contents entry
\renewcommand*\glossarypreamble{\addcontentsline{toc}{chapter}{\glossaryname}}
\setlength{\glspagelistwidth}{0.1\linewidth}
\setlength\glsdescwidth{0.7\linewidth}        % sets width of descriptions
\glsaddall                                    % add all glossaries
\glossarystyle{long}                          % sets glossary style to table
\renewcommand*{\glsgroupskip}{}               % remove vertical space between groups
\printglossary[type=main]                     % print glossaries

