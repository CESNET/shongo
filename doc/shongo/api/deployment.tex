\chapter{Deployment}

This chapter describes the deployment of \gls{g:shongo} to a new \gls{g:domain}.

\section{Applications}

The \gls{g:shongo} consists of the following applications:
\begin{enumerate}
\item \textbf{\Gls{g:controller}} represents an command-line application which should be launched in single instance for each \gls{g:domain} and it acts as the main \gls{g:shongo} application for the \gls{g:domain}.
\item \textbf{\Gls{g:connector}} represents an command-line application which can be launched in multiple instances for each \gls{g:domain} and each instance can manage one or multiple \glspl{g:resource}.
\item \textbf{Client} is command-line interface to a \gls{g:controller} which can be used to setup \gls{g:resource} database and to create \glspl{g:reservation-request}. Multiple instances of \glspl{g:controller-client} can run at the same time.
\end{enumerate}

\section{Installation}

You must install \gls{g:shongo} to each machine where you want to launch a \textbf{\gls{g:controller}}, \textbf{\gls{g:connector}} or \textbf{client}. To install \gls{g:shongo} you need to get the source code. To get the \gls{g:shongo} source code you need to have \code{git}\footnote{Git fast version control \url{http://git-scm.com/}} installed and use the following command:
\begin{verbatim}
git clone username@homeproj.cesnet.cz:shongo
\end{verbatim}
To get an username and password ask at email \href{mailto:martin.srom@cesnet.cz}{\texttt{martin.srom@cesnet.cz}}. 

\subsection{Controller/connector}
To build and launch \textbf{\gls{g:controller}} or \textbf{\gls{g:connector}} you need to have Java Platform (JDK)\footnote{Java Platform (JDK) \url{http://www.oracle.com/technetwork/java/}} and \code{maven}\footnote{Apache Maven Project \url{http://maven.apache.org/download.html}} installed (preferred Maven version is 2.2.1). Enter the following directory:
\begin{verbatim}
cd <repository>/sw/shongo
\end{verbatim}
And type the following command:
\begin{verbatim}
mvn package
\end{verbatim}
The \gls{g:shongo} should be successfully built and tested. 

\subsection{Client}
To launch \textbf{client} you need to have \code{perl}\footnote{Perl \url{http://www.perl.org/get.html}} installed and also the following perl modules:
\begin{compactenum}
\item RPC::XML
\item XML::Twig
\item Text::Table
\item DateTime::Format::ISO8601
\end{compactenum}
On Ubuntu/Debian system, \code{perl} is installed by default and the modules
can be installed by the following command:
\begin{verbatim}
sudo apt-get install librpc-xml-perl libxml-twig-perl \
        libtext-table-perl libdatetime-format-iso8601-perl
\end{verbatim}
All applications (\textbf{\gls{g:controller}}, \textbf{\gls{g:connector}} or \textbf{client}) can be launched by entering the following directory:
\begin{verbatim}
cd <repository>/sw/shongo
\end{verbatim}
And type the \code{./<application>.sh} command:
\begin{verbatim}
./controller.sh
./connector.sh
./client.sh
\end{verbatim}

\section{Controller}
\todo{describe controller configuration}

\section{Connector}
\todo{describe connector configuration}

\section{Client}
\todo{describe client arguments and shell commands}

