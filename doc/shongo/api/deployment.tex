\chapter{Deployment}

This chapter describes the deployment of \gls{g:shongo} to a new \gls{g:domain}.

\section{Applications}

\Gls{g:shongo} consists of the following applications:
\begin{enumerate}
\item \textbf{\Gls{g:controller}} represents a command-line application which should be launched in single instance for each \gls{g:domain} and it acts as the main \gls{g:shongo} application for the \gls{g:domain}.
\item \textbf{\Gls{g:connector}} represents a command-line application which can be launched in multiple instances for each \gls{g:domain} and each instance can manage one or multiple \glspl{g:resource}.
\item \textbf{Client CLI} is command-line interface to a \gls{g:controller} which can be used to setup \gls{g:resource} database and to create \glspl{g:reservation-request}. Multiple instances of \glspl{g:controller-client} can run at the same time.
\item \textbf{Client WEB} \todo{describe web client}
\end{enumerate}

\section{Installation}

You must install \gls{g:shongo} to each machine where you want to launch a \textbf{\gls{g:controller}}, \textbf{\gls{g:connector}} or \textbf{client}. To install \gls{g:shongo} you need to get the source code. To get the \gls{g:shongo} source code you need to have Git\footnote{Git fast version control \url{http://git-scm.com/}} installed and use the following command:
\begin{verbatim}
git clone username@homeproj.cesnet.cz:shongo
\end{verbatim}
To get an username and password ask at \href{mailto:martin.srom@cesnet.cz}{\texttt{martin.srom@cesnet.cz}}. 

\subsection{Controller and connector}
To build and launch \textbf{\gls{g:controller}} or \textbf{\gls{g:connector}} you need to have Java Platform (JDK)\footnote{Java Platform (JDK) \url{http://www.oracle.com/technetwork/java/}} and Maven\footnote{Apache Maven Project \url{http://maven.apache.org/download.html}} installed (preferred Maven version is 2.2.1). Enter the following directory:
\begin{verbatim}
cd <repository>/sw/shongo
\end{verbatim}
And type the following command:
\begin{verbatim}
mvn package
\end{verbatim}
\textbf{Controller} and \textbf{connector} should be successfully built and tested. 

\subsection{Client CLI}
To launch \textbf{command-line client} you need to have Perl\footnote{Perl \url{http://www.perl.org/get.html}} installed and also the following perl modules:
\begin{compactenum}
\item RPC::XML
\item XML::Twig
\item Text::Table
\item DateTime::Format::ISO8601
\item JSON
\item LWP::Protocol::https
\end{compactenum}
On Ubuntu/Debian system, Perl is installed by default and the modules
can be installed by the following command:
\begin{verbatim}
sudo apt-get install librpc-xml-perl libxml-twig-perl \
        libtext-table-perl libdatetime-format-iso8601-perl libjson-perl \
        liblwp-protocol-https-perl
\end{verbatim}
All applications (\textbf{\gls{g:controller}}, \textbf{\gls{g:connector}} or \textbf{client}) can be launched by entering the following directory:
\begin{verbatim}
cd <repository>/sw/shongo
\end{verbatim}
And type the \code{./<application>.sh} command:
\begin{verbatim}
./controller.sh
./connector.sh
./client-cli.sh
\end{verbatim}

\section{Controller}
Controller by default runs on \codeValue{localhost} interface (\codeValue{127.0.0.1}) with XML-RPC server on port \codeValue{8181} and Jade middle-ware on port \codeValue{8282}. To change the default settings command-line attributes can be used:
\begin{verbatim}
./controller.sh --host <host> --jade-port <jade-port> --rpc-port <port>
\end{verbatim}
Another way to change the default settings is to use a configuration file which has more options. Create file \codeValue{<repository>/sw/shongo/controller.cfg.xml} which should contain:
\begin{verbatim}
<?xml version="1.0" encoding="UTF-8" ?>
<configuration>
    ...
</configuration>
\end{verbatim}
The following options are available in the configuration file:
\begin{itemize}
\item Configuration of controlled \gls{g:domain}. It's code \code{name} and name of \code{organization} which run's the
controller.
\begin{verbatim}
<domain>
     <name>cz.cesnet</name>
     <organization>CESNET, z.s.p.o.</organization>
</domain>
\end{verbatim}

\item Configuration of XML-RPC (\code{host} and \code{port}).
\begin{verbatim}
<rpc>
    <host>127.0.0.1</host>
    <port>8181</port>
</rpc>
\end{verbatim}

\item Configuration of XML-RPC authentication and authorization. An \code{authorization-server} can be specified.
\begin{verbatim}
<security>
    <authorization-server>
        https://hroch.cesnet.cz/phpid-server/oic/
    </authorization-server>
</security>
\end{verbatim}

\item Configuration of Jade middle-ware (\code{host} and \code{port}).
\begin{verbatim}
<jade>
    <host>127.0.0.1</host>
    <port>8282</port>
</jade>
\end{verbatim}

\item Configuration of \glspl{g:reservation}. Maximum duration of \gls{g:resource} \gls{g:reservation} and \gls{g:device-alias} \gls{g:reservation} can be configured.
\begin{verbatim}
<reservation>
    <resource>
        <max-duration>P6D</max-duration>
    </resource>
    <alias>
        <max-duration>P1Y</max-duration>
    </alias>
</reservation>
\end{verbatim}

\item Configuration of \gls{g:controller}'s worker which periodically runs \gls{g:scheduler} and \gls{g:preprocessor}. A \code{period} in which the worker runs and the length of a working \code{interval} can be configured. The \codeValue{PT10S} period means that every 10 seconds \gls{g:scheduler} and \gls{g:preprocessor} is executed. The \codeValue{P31D} interval means that \gls{g:preprocessor} and \gls{g:scheduler} will process only \glspl{g:reservation-request} which are 31 days ahead.
\begin{verbatim}
<worker>
    <period>PT10S</period>
    <interval>P31D</interval>
</worker>
\end{verbatim}

\item Configuration of \gls{g:executor} which periodically checks for allocated \glspl{g:executable} (e.g., \glspl{g:compartment}) and executes them. A \code{period} can be configured in which the executor performs checking and also the \code{lookup-ahead} which is used when searching for allocated \glspl{g:compartment} to be executed (the executor executes only \glspl{g:compartment} which take place in interval starting at current date/time and with length of \code{lookup-ahead}). Option \code{start} defines duration by which is modified the starting date/time of the executed \gls{g:executable} and the \code{end} modifies the ending date/time respectively (both \codeValue{PT-30S} means that the \gls{g:executable} will be started and ended 30 seconds beforehand). Periods \code{waiting-start} and \code{waiting-end} specify how often the \gls{g:executor} checks whether the \gls{g:executable} should actually be started or stopped. Option \code{waiting-virtual-room} specifies how long should the \gls{g:executor} wait after it send commands for creating all \glspl{g:device-virtual-room} in a \gls{g:compartment} before it connects all allocated endpoints to proper \glspl{g:device-virtual-room}.
\begin{verbatim}
<executor>
    <period>PT15S</period>
    <lookup-ahead>PT3M</lookup-ahead>
    <executable>
        <start>PT-30S</start>
        <end>PT-30S</end>
        <waiting-start>PT10S</waiting-start>
        <waiting-end>PT10S</waiting-end>
    </executable>
    <compartment>
        <waiting-virtual-room>PT10S</waiting-virtual-room>
    </compartment>
</executor>
\end{verbatim}
\end{itemize}

\section{Connector}
Connector by default connects to the controller on \codeValue{127.0.0.1:8282} through Jade middle-ware and the connector runs Jade middle-ware on \codeValue{127.0.0.1:8383}. To change the default settings command-line attributes can be used:
\begin{verbatim}
./connector.sh --controller <host>:<port> --host <jade-host> --port <jade-port>
\end{verbatim}
Another way to change the default settings is to use a configuration file which has more options. Create file \codeValue{<repository>/sw/shongo/connector.cfg.xml} which should contain:
\begin{verbatim}
<?xml version="1.0" encoding="UTF-8" ?>
<configuration>
    ...
</configuration>
\end{verbatim}
The following options are available in the configuration file:
\begin{itemize}
\item Configuration of the connection to the controller (\code{host} and \code{port}). The connection to the controller is periodically checked and in case of failure the connection is re-established. The option \code{connection-check-period} specifies how often the check is performed.
\begin{verbatim}
<controller>
    <host>127.0.0.1</host>
    <port>8282</port>
    <connection-check-period>PT10S</connection-check-period>
</controller>
\end{verbatim}

\item Configuration of Jade middle-ware (\code{host} and \code{port}).
\begin{verbatim}
<jade>
    <host>127.0.0.1</host>
    <port>8383</port>
</jade>
\end{verbatim}

\item Configuration of \gls{g:connector-agent} instances which are automatically started when the \gls{g:connector} is launched. Each \gls{g:connector} can run multiple \glspl{g:connector-agent}.
\begin{verbatim}
<instances>
    <instance>
        ...
    </instance>
    <instance>
        ...
    </instance>
    ...
</instances>
\end{verbatim}

Example of \gls{g:connector-agent} configuration for Tandberg Codec C90:
\small\begin{verbatim}
<instance>
    <name>c90</name>
    <device>
        <connectorClass>cz.cesnet.shongo.connector.CodecC90Connector</connectorClass>
        <host>147.251.54.102</host>
        <port>22</port>
        <auth>
            <username>(username)</username>
            <password>(password)</password>
        </auth>
    </device>
</instance>
\end{verbatim}\normalsize

Example of \gls{g:connector-agent} configuration for Codian MCU 4515:
\small\begin{verbatim}
<instance>
    <name>mcu</name>
    <device>
        <connectorClass>cz.cesnet.shongo.connector.CiscoMCUConnector</connectorClass>
        <host>mcuc.cesnet.cz</host>
        <auth>
            <username>(username)</username>
            <password>(password)</password>
        </auth>
    </device>
</instance>
\end{verbatim}\normalsize
\end{itemize}


\section{Client CLI}
Command-line client can be started and connected to a \gls{g:controller} by the following commands:
\begin{verbatim}
./client-cli.sh --connect 127.0.0.1 [--testing-access-token]
./client-cli.sh --connect 127.0.0.1:8181 [--testing-access-token]
\end{verbatim}
Option \codeValue{----testing-access-token} forces the client to skip the user authentication and to use a testing access token which is automatically validated in the \gls{g:controller}.
The available commands in the client shell can be displayed by typing:
\begin{verbatim}
shongo> <TAB>
\end{verbatim}
Or by typing:
\begin{verbatim}
shongo> help <ENTER>
\end{verbatim}


