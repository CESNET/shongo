\appendix

\definecolor{ApiLstKeywordColor}{rgb}{0,0,0.4}
\lstset{
  extendedchars=false,
  alsoletter={-},
  basicstyle=\small\fontfamily{txtt}\fontseries{b}\selectfont,
  keywordstyle=\color{ApiLstKeywordColor}\selectfont,
  commentstyle=\color{black!60}\selectfont,
}
\lstdefinestyle{ApiLstStyle}{
  language=java,
  morekeywords={enum, atomic_type},
  breaklines=true,
  breakatwhitespace=true,
  breakautoindent=true,
}

\providecounter{ApiCounter}
\providecommand{\ApiLabel}[1]{\refstepcounter{ApiCounter}\label{#1}}

%%%%%%%%%%%%%%%%%%%%%%%%%%%%%%%%%%%%%%%%%%%%%%%%%%%%%%%%%%%%%%%%%%%%%%%%%%%%%%%%
% API definition
\provideenvironment{Api}{\begin{itemize}}{\end{itemize}}

%%%%%%%%%%%%%%%%%%%%%%%%%%%%%%%%%%%%%%%%%%%%%%%%%%%%%%%%%%%%%%%%%%%%%%%%%%%%%%%%
% API definition code and value
\providecommand{\ApiCode}[1]{\lstinline[style=ApiLstStyle]{#1}}
\providecommand{\ApiValue}[1]{\texttt{#1}}
% \end{verbatim}
% Previous line corrects syntax coloring parser

%%%%%%%%%%%%%%%%%%%%%%%%%%%%%%%%%%%%%%%%%%%%%%%%%%%%%%%%%%%%%%%%%%%%%%%%%%%%%%%%
% API definition item
\providecommand{\ApiItem}[1]{\item #1 %

% Previous empty line is intended to be blank (wraps the text
% to the next line even if the \ApiItem is not followed by empty line)
}

%%%%%%%%%%%%%%%%%%%%%%%%%%%%%%%%%%%%%%%%%%%%%%%%%%%%%%%%%%%%%%%%%%%%%%%%%%%%%%%%
% API command
\provideenvironment{ApiCmd}[2]{{\item \ApiRef{#1} \ApiCode{#2(}}%
\def\ApiCmdTmpFirst{\boolean{true}}%
}{\ApiCode{)}%

% Previous empty line is intended to be blank
}
\provideenvironment{ApiCmdCollection}[3]{{\item \ApiCode{#1<}\ApiRef{#2}\ApiCode{>} \ApiCode{#3(}}%
\def\ApiCmdTmpFirst{\boolean{true}}%
}{\ApiCode{)}%

% Previous empty line is intended to be blank
}
\providecommand{\ApiCmdParam}[2]{%
\ifthenelse{\ApiCmdTmpFirst}{}{, }%
\ApiRef{#1} \ApiCode{#2}%
\def\ApiCmdTmpFirst{\boolean{false}}%
}

%%%%%%%%%%%%%%%%%%%%%%%%%%%%%%%%%%%%%%%%%%%%%%%%%%%%%%%%%%%%%%%%%%%%%%%%%%%%%%%%
% API atomic type
\providecommand{\ApiType}[2]{\ApiLabel{api:#1}\ApiItem{%
  \ifx&#2& \ApiCode{atomic_type #1} \else \ApiCode{atomic_type #1 =\ }\ApiRef{#2} \fi}%
}

%%%%%%%%%%%%%%%%%%%%%%%%%%%%%%%%%%%%%%%%%%%%%%%%%%%%%%%%%%%%%%%%%%%%%%%%%%%%%%%%
% API class
\providecommand{\ApiClass}[2]{\ApiLabel{api:#1}\ApiItem{%
  \ifx&#2& \ApiCode{class #1} \else \ApiCode{class #1 extends\ }\ApiRef{#2} \fi}%
}

%%%%%%%%%%%%%%%%%%%%%%%%%%%%%%%%%%%%%%%%%%%%%%%%%%%%%%%%%%%%%%%%%%%%%%%%%%%%%%%%
% API class attributes
\provideenvironment{ApiClassAttributes}{% Next empty line is intended to be blank

\begin{samepage}\textbf{Attributes:}\begin{compactitem}}{\end{compactitem}\end{samepage}}
\providecommand{\ApiRequired}{{\color{blue!50!black}\textbf{Required}}}
\providecommand{\ApiOptional}{{\color{gray}\textbf{Optional}}}
\providecommand{\ApiOptionalDefault}[1]{{\color{gray}\textbf{Optional}, default: \ApiValue{#1}}}
\providecommand{\ApiReadOnly}{{\color{red!50!black}\textbf{ReadOnly}}}
\providecommand{\ApiClassAttribute}[3]{\ApiItem{\ApiRef{#2} \ApiCode{#1} \hspace{1mm}(\ifx&#3&\ApiReadOnly\else#3\fi)}}
\providecommand{\ApiClassAttributeCollection}[4]{\ApiItem{\ApiCode{#2<}\ApiRef{#3}\ApiCode{>} \ApiCode{#1} \hspace{1mm}(\ifx&#4&\ApiReadOnly\else#4\fi)}}

%%%%%%%%%%%%%%%%%%%%%%%%%%%%%%%%%%%%%%%%%%%%%%%%%%%%%%%%%%%%%%%%%%%%%%%%%%%%%%%%
% API enum
\providecommand{\ApiEnum}[1]{\ApiLabel{api:#1}\ApiItem{\ApiCode{enum #1}}}
\provideenvironment{ApiEnumValues}{% Next empty line is intended to be blank

\begin{samepage}\textbf{Enumeration values:}\begin{compactitem}}{\end{compactitem}\end{samepage}}
\providecommand{\ApiEnumValue}[2]{\ApiItem{{\ApiCode{#1} \ifx&#2& \else \ApiValue{(#2)} \fi}}}

%%%%%%%%%%%%%%%%%%%%%%%%%%%%%%%%%%%%%%%%%%%%%%%%%%%%%%%%%%%%%%%%%%%%%%%%%%%%%%%%
% API example
\providecommand{\ApiExample}{% Next empty line is intended to be blank

\textbf{Example:}
}

%%%%%%%%%%%%%%%%%%%%%%%%%%%%%%%%%%%%%%%%%%%%%%%%%%%%%%%%%%%%%%%%%%%%%%%%%%%%%%%%
% API note
\providecommand{\ApiNote}{% Next empty line is intended to be blank

\textbf{Note:}
}

%%%%%%%%%%%%%%%%%%%%%%%%%%%%%%%%%%%%%%%%%%%%%%%%%%%%%%%%%%%%%%%%%%%%%%%%%%%%%%%%
% API failures
\provideenvironment{ApiFailures}{\begin{compactitem}}{\end{compactitem}}
\providecommand{\ApiFailure}[1]{\ApiItem{\ApiCode{faultCode = #1}}}

%%%%%%%%%%%%%%%%%%%%%%%%%%%%%%%%%%%%%%%%%%%%%%%%%%%%%%%%%%%%%%%%%%%%%%%%%%%%%%%%
% API ref
\providecommand{\ApiRef}[1]{%
\foreach \ApiTmp [count=\ApiTmpIndex] in{#1}{%
%
\ifthenelse{\ApiTmpIndex > 1}{\ApiCode{|}}{}%
%
\def\tmpResult{\boolean{false}}%
\ifthenelse{\equal{\ApiTmp}{String}}{\def\tmpResult{\boolean{true}}}{}%
\ifthenelse{\equal{\ApiTmp}{void}}{\def\tmpResult{\boolean{true}}}{}%
\ifthenelse{\equal{\ApiTmp}{int}}{\def\tmpResult{\boolean{true}}}{}%
\ifthenelse{\equal{\ApiTmp}{boolean}}{\def\tmpResult{\boolean{true}}}{}%
\ifthenelse{\equal{\ApiTmp}{long}}{\def\tmpResult{\boolean{true}}}{}%
\ifthenelse{\equal{\ApiTmp}{float}}{\def\tmpResult{\boolean{true}}}{}%
\ifthenelse{\equal{\ApiTmp}{byte[]}}{\def\tmpResult{\boolean{true}}}{}%
% 
\ifthenelse{\tmpResult}{%
\textbf{\texttt{\ApiTmp}}
}{%
\hyperref[api:\ApiTmp]{\code{\ApiTmp}}%
}%  
}%
}


\chapter{Client Usage}

Client can be started and connected to a \gls{g:controller} by the following commands:
\begin{verbatim}
./client-cli.sh --connect localhost --testing-access-token
\end{verbatim}
New \gls{g:resource} can be interactively created by typing:
\begin{verbatim}
shongo> create-resource
\end{verbatim}
New \gls{g:resource} can be automatically created by typing:
\begin{verbatim}
shongo> create-resource -confirm { \
    class: 'DeviceResource', \
    name: 'mcu', \
    allocatable: 1, \
    technologies: ['H323'], \
    mode: { \
        connectorAgentName: 'mcu' \
    }, \
    capabilities: [{ \
        class: 'VirtualRoomsCapability', \
        portCount: 100 \
    }, { \
        class: 'AliasProviderCapability', \
        technology: 'H323', \
        type: 'E164', \
        patterns: ['9500872[dd]'], \
        restrictedToOwnerResource: 1 \
    }] \
}
\end{verbatim}
List of existing \glspl{g:resource} can be showed by typing:
\begin{verbatim}
shongo> list-resources
\end{verbatim}
Detail of existing \gls{g:resource} can be showed by typing:
\begin{verbatim}
shongo> get-resource <resource-identifier>
\end{verbatim}
Summary of allocation for existing \gls{g:resource} can be showed by typing:
\begin{verbatim}
shongo> get-resource-allocation <resource-identifier> [-interval 2012-01-01/P1Y]
\end{verbatim}
New \gls{g:reservation-request} can be interactively created by typing:
\begin{verbatim}
shongo> create-reservation-request
\end{verbatim}
New \gls{g:reservation-request} can be automatically created by typing:
\begin{verbatim}
shongo> create-reservation-request -confirm { \
    class: 'PermanentReservationRequest', \
    name: 'Example', \
    resourceIdentifier: 'shongo:cz.cesnet:1', \
    slots: [{ \
        start: '2012-01-01T12:00', \
        duration: 'PT4M' \
    }] \
}
\end{verbatim}
List of existing \glspl{g:reservation-request} can be showed by typing:
\begin{verbatim}
shongo> list-reservatin-requests
\end{verbatim}
Detail of existing \gls{g:reservation-request} can be showed by typing:
\begin{verbatim}
shongo> get-reservation-request <reservation-request-identifier>
\end{verbatim}
Allocated \gls{g:reservation}(s) for existing \gls{g:reservation-request} can be showed by typing:
\begin{verbatim}
shongo> get-reservation-for-request <reservation-request-identifier>
\end{verbatim}
Or by typing:
\begin{verbatim}
shongo> get-reservation <reservation-identifier>
\end{verbatim}
For scripting purposes it is useful to run the client with command(s) which should be executed (the client will not run the shell in this case and exits immediately):
\begin{verbatim}
./client-cli.sh --connect localhost --testing-access-token \
    --cmd "list-resources" \
    --cmd "list-reservation-requests"
\end{verbatim}

\chapter{Controller API Usage}

\section{Perl programming language}
\newenvironment{PerlCmd}{\small\verbatim}{\endverbatim}
\newenvironment{PerlResponse}{\textbf{Response}\small\verbatim}{\endverbatim}

\subsection{Connect to Controller}

\begin{PerlCmd}
#!/usr/bin/perl

require RPC::XML;
require RPC::XML::Client;

$client = RPC::XML::Client->new('http://localhost:8008');

$response = $client->send_request(...);

if ( ref($response) ) {
    use XML::Twig;
    $xml = XML::Twig->new(pretty_print => 'indented');
    $xml->parse($response->as_string());
    $xml->print();
} else {
    print($response . "\n");
}
\end{PerlCmd}

\newpage
\subsection{Create reservation}
\begin{PerlCmd}
$response = $client->send_request(
    'Reservation.createReservationRequest',
    RPC::XML::string->new('1e3f174ceaa8e515721b989b19f71727060d0839'), # access token
    RPC::XML::struct->new(
        'class' => RPC::XML::string->new('ReservationRequest'),
        'slot' => RPC::XML::string->new('20120101T12:00/PT2H'),
        'name' => RPC::XML::string->new('test'),
        'purpose' => RPC::XML::string->new('EDUCATION')
        ...
    )
);
\end{PerlCmd}
\begin{PerlResponse}
<struct>
  <member>
    <name>class</name>
    <value><string>ReservationRequest</string></value>
  </member>
  <member>
    <name>id</name>
    <value>
      <string>shongo:cz.cesnet:1</string>
    </value>
  </member>
  <member>
    <name>slot</name>
    <value><string>20120101T12:00:00/PT2H</string></value>
  </member>
  ...
</struct>
\end{PerlResponse}

\newpage
\subsection{Modify reservation}
\begin{PerlCmd}
$response = $client->send_request(
    'Reservation.modifyReservationRequest',
    RPC::XML::string->new('1e3f174ceaa8e515721b989b19f71727060d0839'),
    RPC::XML::string->new('shongo:cz.cesnet:1'),
    RPC::XML::struct->new(
        'description' => RPC::XML::struct->new() # set description to null
    )
);
\end{PerlCmd}
\begin{PerlResponse}
<struct>
  <member>
    <name>id</name>
    <value><string>shongo:cz.cesnet:1</string></value>
  </member>
  <member>
    <name>class</name>
    <value><string>ReservationRequest</string></value>
  </member>
  <member>
    <name>type</name>
    <value><string>20120101T12:00:00/PT2H</string></value>
  </member>
  ...
</struct>
\end{PerlResponse}

\newpage
\subsection{List reservations}
\begin{PerlCmd}
$response = $client->send_request(
    'Reservations.listReservationRequests',
    RPC::XML::string->new('1e3f174ceaa8e515721b989b19f71727060d0839')
);
\end{PerlCmd}
\begin{PerlResponse}
<array><data>
  <value><struct>
    <member>
      <name>id</name>
      <value><string>shongo:cz.cesnet:1</string></value>
    </member>
    <member>
      <name>class</name>
      <value><string>ReservationRequest</string></value>
    </member>
    <member>
      <name>type</name>
      <value><string>20120101T12:00:00/PT2H</string></value>
    </member>
    ...
  </struct></value>
</data></array>
\end{PerlResponse}

\newpage
\subsection{Exception handling}
\subsubsection{Wrong class}
\begin{PerlCmd}
$response = $client->send_request(
    'Reservations.listReservationRequests',
    RPC::XML::string->new('1e3f174ceaa8e515721b989b19f71727060d0839')
);
\end{PerlCmd}
\begin{PerlResponse}
<fault>
  <value><struct>
    <member>
      <name>faultString</name>
      <value><string>Class 'SecurityTokenX' is not defined.</string></value>
    </member>
    <member>
      <name>faultCode</name>
      <value><i4>10</i4></value>
    </member>
  </struct></value>
</fault>
\end{PerlResponse}

\subsubsection{Wrong attribute name}
\begin{PerlCmd}
$response = $client->send_request(
    'Reservations.createReservationRequest',
    RPC::XML::string->new('1e3f174ceaa8e515721b989b19f71727060d0839'),
    RPC::XML::struct->new(
        'typeX' => RPC::XML::string->new('PERMANENT')
    )
);
\end{PerlCmd}
\begin{PerlResponse}
<fault>
  <value><struct>
    <member>
      <name>faultString</name>
      <value><string>Attribute 'typeX' in class 'Reservation' is not defined.</string></value>
    </member>
    <member>
      <name>faultCode</name>
      <value><i4>12</i4></value>
    </member>
  </struct></value>
</fault>
\end{PerlResponse}

\subsubsection{Wrong attribute value}
\begin{PerlCmd}
$response = $client->send_request(
    'Reservations.createReservationRequest',
    RPC::XML::string->new('1e3f174ceaa8e515721b989b19f71727060d0839'),
    RPC::XML::struct->new(
        'purpose' => RPC::XML::struct->new(
            'class' => RPC::XML::string->new('SecurityToken')
        )
    )
);
\end{PerlCmd}
\begin{PerlResponse}
<fault>
  <value><struct>
    <member>
      <name>faultString</name>
      <value><string>Attribute 'purpose' in class 'ReservationRequest' has type
          'ReservationRequestPurpose' but 'SecurityToken' was presented.</string></value>
    </member>
    <member>
      <name>faultCode</name>
      <value><i4>13</i4></value>
    </member>
  </struct></value>
</fault>
\end{PerlResponse}

\subsubsection{Wrong enum}
\begin{PerlCmd}
$response = $client->send_request(
    'Reservations.createReservationRequest',
    RPC::XML::string->new('1e3f174ceaa8e515721b989b19f71727060d0839'),
    RPC::XML::struct->new(
        'purpose' => RPC::XML::string->new('SCIENCEX')
    )
);
\end{PerlCmd}
\begin{PerlResponse}
<fault>
  <value><struct>
    <member>
      <name>faultString</name>
      <value><string>Enum value 'SCIENCEX' is not defined in enum
          'ReservationRequestPurpose'.</string></value>
    </member>
    <member>
      <name>faultCode</name>
      <value><i4>20</i4></value>
    </member>
  </struct></value>
</fault>
\end{PerlResponse}

\chapter{JADE Command Encoding Example} \label{appendix:jade-command-encoding}

Consider the following command required by this API:
\begin{Api}
\ApiItem{\ApiCode{List<RoomUser> listParticipants(SecurityToken token, String roomId)}}
\end{Api}
The following classes should be defined to represent the command and all objects used by it:
\begin{verbatim}
package cz.cesnet.shongo.jade.ontology;

public class ListParticipants implements AgentAction {
    private String roomId;

    public String getRoomId() {
        return roomId;
    }
    public void setRoomId(String roomId) {
        this.roomId = roomId;
    }
}

public class UserIdentity implements Concept {
    private String id;

    public String getId() {
        return id;
    }
    public void setId(String id) {
        this.id = id;
    }
}

public class RoomUser implements Concept {
    private String userId;
    private String roomId;
    private UserIdentity userIdentity;
    private boolean muted;
    private int microphoneLevel;
    private int playbackLevel;

    // getters and setters ...
}
\end{verbatim}

The command might be encoded in the following message:
\begin{verbatim}
(REQUEST
 :sender  ( agent-identifier :name Controller@Shongo :addresses
   (sequence http://127.0.0.1:7778/acc http://127.0.0.1:49879/acc ))
 :receiver  (set ( agent-identifier :name mcu@Shongo ) )
 :content  "((action (agent-identifier :name Controller@Shongo :addresses
   (sequence http://127.0.0.1:7778/acc http://127.0.0.1:49879/acc)) (ListParticipants :
roomId shongo-test)))"
 :language  fipa-sl  :ontology  shongo-ontology  :protocol  fipa-request
)
\end{verbatim}

A successful reply would then be encoded as follows:
\begin{verbatim}
(INFORM
 :sender ( agent-identifier :name mcu@Shongo  :addresses (sequence
           http://127.0.0.1:7778/acc http://127.0.0.1:49879/acc ))
 :receiver (set ( agent-identifier :name Controller@Shongo  :addresses
                  (sequence http://127.0.0.1:7778/acc http://127.0.0.1:49879/acc)
                 ) )
 :content
     "((result (action (agent-identifier :name Controller@Shongo :addresses
      (sequence http://127.0.0.1:7778/acc http://127.0.0.1:49879/acc))
      (ListParticipants :roomId shongo-test)) (serializable :value
      rO0ABXNyABNqYXZhLnV0aWwuQXJyYXlMaXN0eIHSHZnHYZ0DAAFJAARzaXpleHAAAAABdw
      ...
      wyGuMDAAB4cHcPAA1FdXJvcGUvUHJhZ3VleHhwdAALc2hvbmdvLXRlc3R0AAQzMzQ1cHg=
 )))"
 :reply-with  Controller@Shongo1351726810139  :language  fipa-sl
 :ontology  shongo-ontology  :protocol  fipa-request
 :conversation-id  C10101047_1351726810116
)
\end{verbatim}
Note that the result value is a serialized Java collection.
