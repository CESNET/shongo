\documentclass[a4paper]{report}
\usepackage{geometry}
\geometry{paper=a4paper}

\usepackage[T1]{fontenc}
\usepackage[utf8]{inputenx}
\usepackage{palatino}
\usepackage{mathpazo}
\usepackage{microtype}
\renewcommand*\ttdefault{txtt}
\usepackage{eurosym}

\usepackage[czech,english]{babel}

\usepackage[pdftex,breaklinks=true,pdfborder={0 0 0}]{hyperref}
\usepackage{tocbibind}
\usepackage{url}
\usepackage{paralist}
\usepackage{graphicx}
\usepackage{xtab}
\usepackage{booktabs}
\usepackage{calc}
\usepackage{ifthen}
\usepackage{xspace}

\newenvironment{metrics}{\par\emph{Metrics:}\begin{compactitem}}{\end{compactitem}}

\title{Analysis of Available Discovery and Messaging Systems}
\author{Petr Holub, Martin Šrom, Ondřej Bouda}
\date{2012}

\begin{document}

\maketitle

\tableofcontents


\chapter{Introduction}



\chapter{Requirements}

\section{Features}

\begin{itemize}

\item Multiple agents per single JVM

\item Types of messages:

\begin{compactitem}

\item modes: unicast, multicast, broadcast, anycast

\item reliability: reliable messaging, low-overhead unreliable messaging

\end{compactitem}

\begin{metrics}

\item yes/no for each distribution and reliability mode

\end{metrics}

\item Firewall and NAT traversal

\begin{itemize}

\item is firewall traversal supported?

\item what types of NAT can be traversed?

\item what ports are used?

\end{itemize}

\item Resource discovery

\begin{itemize}

\item how can we list all the agents/entities in the domain? is there any discovery service?

\end{itemize}


\end{itemize}



\section{Scalability}

\begin{itemize}

\item Scalability with respect to the number of entities within the domain.

There can be thousands of agents controlling various devices. If organized in
classical agent-based systems, this can be implemented as per-device agend and
thus we need to know how much resources is this going to consume the middleware
per se. If only one instance per node is available, the agents need to be
implemented on higher level, but we still need to know resource overhead at
least per each instance.

\begin{metrics}

\item CPU load for 1, 10, 100, 1.000 clients per node

\item memory used for 1, 10, 100, 1.000 instances per node

\end{metrics}


\item Overhead per message.

This is important especially when sending small messages.

\begin{metrics}

\item header size per message

\end{metrics}



\item Scalability with respect to the number of messages sent.

There may be a large number of messages sent between agent pairs, especially
for various monitoring reasons.

\begin{metrics}

\item CPU load for sending 10, 100, 1.000, 10.000 messages per second

\item maximum number of messages per second on defined configuration

\end{metrics}

\end{itemize}



\section{Security}

The system will run in a decentralized environment where federated
authentication are the most appropriate among currently implemented security
concepts. This comprises several questions:

\begin{itemize}

\item authentication of entities within an administrative domain

\item authentication of entities among administrative domains

\item use with Shibboleth

\item use with X.509 certificates

\item encryption of messages

\item performance overhead of authentication and encryption

\end{itemize}

\begin{metrics}

\item list of authentication options for inter- and intra-domain modes

\item CPU load for 1, 10, 100, 1.000, 10.000 messages per second for
performance overhead, and message size overhead

\item yes/no other questions

\end{metrics}



\section{Robustness}

Robustness with respect to outages of any component in the system is very
important due to distributed nature of the system.

\begin{itemize}

\item Is it available?

\item What is the overhead of robustness?

\item What is time to recover after various types of failures in the system?

\item Messaging performance over lossy networks.

\item CPU load for 1, 10, 100, 1.000, 10.000 messages per second for
performance overhead, and message size overhead

\end{itemize}

\begin{metrics}

\item yes/no for availibility

\item (currently don't have exact metric for overhead -- needs to be identified
based on options available)

\item time to recover after various node and service outages

\item maximum number of messages per second on defined configuration

\end{metrics}







\chapter{Description of Evaluated Systems}

\section{JXTA}

\paragraph{Description.}

A substrate for peer-to-peer applications, available in Java (JXSE) and C
flavors (JXTA-C).

Currently versions 2.6 and 2.7 are available and the development has
significantly slowed down after Sun was bought by Oracle.  Version 2.6 seems to
be used for most of the projects now as 2.7 was orphaned immediately after its
release\footnote{\url{http://jverstry.blogspot.com/2011/04/jxtajxse-27-is-out-but-what-about.html}}.

A few random notes:

\begin{compactitem}

\item Until 2.7, it was impossible to run multiple peers on a singe JVM.

\item Because of stale development, JXTA does not seem to be suitable for
launching a new project.

\end{compactitem}


There is a fork based on 2.7 concepts called Chaupal
(\url{https://code.google.com/p/chaupal/}) lead by J\'{e}r\^{o}me Verstrynge
(one of the JXTA leaders in recent years), but it is far from being anything
usable for production as of writing this report.

\paragraph{Notable deployments.}

\begin{compactitem}

\item CoUniverse

\item onedrum.com (used custom 2.6 branch and later migrated to their own
framework)

\item b2een (Amalto Technologies - http://www.amalto.com, used custom 2.5
branch in 2009--2010, no reports after that)

\item sixearch (last release in 2009)

\item Collanos Workplace (last release in 2009)

\end{compactitem}


\paragraph{License and availability.}

Open source, available for download. Uses The Sun Project JXTA(TM) Software License (Based on the Apache Software License Version 1.1).


\section{JADE}

\paragraph{Description.} An open-source middleware for agent-based programming. Originally developed by TILAB (Telecom Italia), nowadays supervised by the JADE Board (Telecom Italia, Motorola, Whitestein Technologies AG, Profactor GmbH, and France Telecom R\&D). It follows the FIPA specifications\footnote{\url{http://www.fipa.org}}. It may be run on virtually any types of devices supporting Java (the API is the same for the J2EE, J2SE and J2ME).

A JADE platform is composed of several run-time \textsl{containers} launched over one or more hosts across a network. In the platform, there is a special, \textsl{main} container providing the platform services. Backup containers may be run to automatically take over the main container in case of failure. Multiple agents may run in a single container (within a single JVM). Within the platform, agents communicate directly with each other, involving the main container services just for localizing the recipient (which is cached). Agents from multiple platforms may communicate with each other, if permitted. Agent mobility within a platform is supported.

The agents are programmed by means of implementing behaviours. Various behaviour types are offered by the framework (e.g. one-shot, simple repeating, or a behaviour defined by a finite-state automaton) and composing simpler behaviours is also possible. The library even contains complete FIPA standard agent interaction protocol implementations, such as the FIPA-Contract-Net (used for task delegation), or distributed leader election. The content of messages may be parsed and processed utilizing the ontologies and content languages concepts.

The documentation is good and offers plenty of examples of various features. As the last resort, the JADE source code is available (well-structured and easy to read).

There are add-ons available for download, developed by the JADE team or by third parties. The most notable ones include the Security and Trusted Agents add-ons (adding authentication and autorization capabilities), Leap (allowing deployment on mobile devices), or Java-Sniffer.


\paragraph{Notable deployments.} Besides the companies present in the JADE Board (see the description above), some others using JADE are worth mentioning:
\begin{compactitem}
\item British Telecommunications use JADE as the core platform for mPower -- a system for cooperation among mobile workers and team-based job management \cite{bellifemine2008jade};
\item Yuan Ze University, Taiwan developed MADIP -- a health-care monitoring system \cite{su2011jade} (November 2009);
\item Rockwell Automation wrote the Java-Sniffer add-on (October 2006);
\item Logica wrote the Trusted Agents add-on (December 2011);
\item TeliaSonera\footnote{a dominant telephone company and mobile network operator in Sweden and Finland} implemented the CASCOM HTTP MTP add-on for usage in unreliable wireless networks, e.g. in 3G-networks (January 2008);
\item Siemens committed some bugfixes in version 4.0 (April 2010).
\end{compactitem}
Telecom Italia has developed a JADE-based mediation layer NNEM to decouple Operations Support Systems from the network equipments and functionalities \cite{bellifemine2008jade}. It manages about 10,000 devices, while serving more than 100,000 devices is planned.


\paragraph{License and availability.} Licensed under LGPL. Open source, available for download. The development is still active and is likely to go on, since JADE is considered the most popular software agent platform by its authors \cite{bellifemine2008jade}.




\section{Mule ESB}

\paragraph{Description.}

\paragraph{Notable deployments.}

\paragraph{License and availability.}


\section{FUSE ESB}

\paragraph{Description.}

\paragraph{Notable deployments.}

\paragraph{License and availability.}



\section{Petals ESB}

\paragraph{Description.}

\paragraph{Notable deployments.}

\paragraph{License and availability.}


\section{GEMBUS}

\paragraph{Description.}

\paragraph{Notable deployments.}

\paragraph{License and availability.}


\section{FreePastry}

\paragraph{Description.}

\paragraph{Notable deployments.}

\paragraph{License and availability.}


\section{Peer to Peer Simplified (P2PS)}

\paragraph{Description.}

\paragraph{Notable deployments.}

\paragraph{License and availability.}




\chapter{Comparison of Systems}

\chapter{Conclusions}

\bibliographystyle{plain}
\bibliography{analysis}


\end{document}

