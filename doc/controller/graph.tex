\usepackage{ifthen}

% Draw vertex (node, position, title)
\newcommand{\Vertex}[3]{\node[vertex] (#2) at (#1)  {#3};}

% Draw edge (type, styles, from, to)  
\newcounter{EdgeCount}      
\newcommand{\Edge}[4]{
    % Count edges
    \setcounter{EdgeCount}{0}
    \foreach \x in{#2}{
        \stepcounter{EdgeCount}
    }    

    % Determine edges starting bend ratio
    \def\EdgeCountOriginal{\arabic{EdgeCount}}
    \def\EdgeCountEven{\the\numexpr((\value{EdgeCount}-1)/2)*2\relax}    
    \if\EdgeCountOriginal\EdgeCountEven
       % For even edges count
       \def\EdgeBegin{\the\numexpr5-((\value{EdgeCount}-1)/2)*10\relax}
    \else
       % For odd edges count
       \def\EdgeBegin{\the\numexpr0-((\value{EdgeCount}-1)/2)*10\relax}
    \fi

    % Determine edge arrow style    
    \def\EdgeType{-}
    \ifthenelse{\equal{#1}{<->}}{\def\EdgeType{latex-latex}}{}
    \ifthenelse{\equal{#1}{->}}{\def\EdgeType{-latex}}{}
    \ifthenelse{\equal{#1}{<-}}{\def\EdgeType{latex-}}{}
       
    % Draw edges
    \foreach \EdgeStyle [count=\EdgeStyleIndex] in{#2}{
      \draw [
        \EdgeType, 
        edge,
        \EdgeStyle, 
        bend left=\the\numexpr\EdgeBegin+(\EdgeStyleIndex - 1)*10\relax
      ] (#3) edge (#4);
    }
}

% Draw all edges between the nodes (type, styles, all)
\newcommand{\EdgeAllToAll}[3] {
  \foreach \EdgeNodeFrom [count=\EdgeNodeFromIndex] in{#3}{
    \foreach \EdgeNodeTo [count=\EdgeNodeToIndex] in{#3}{
      \ifthenelse{\EdgeNodeToIndex > \EdgeNodeFromIndex}{
        \Edge{#1}{#2}{\EdgeNodeFrom}{\EdgeNodeTo}        
      }{}
    } 
  }
}

% Draw all edges between the nodes (type, styles, one, all)
\newcommand{\EdgeOneToAll}[4] {
  \foreach \EdgeNodeTo in{#4}{
    \Edge{#1}{#2}{#3}{\EdgeNodeTo}
    
  }
}

% Legend (position)
\newenvironment{GraphLegend}[1]{
  \begin{pgfinterruptboundingbox}
  \node (legend) at ($(#1) + (0, 0.25)$) {};
  \end{pgfinterruptboundingbox}
}{
}

% Legend Item (style, title)
\newcommand{\GraphLegendItem}[2]{  
  \node (legend) at ($(legend) + (0, -0.5)$) {};
  \draw [#1, line width=1.5pt]  (legend) --++ (-0.9, 0);
  \node[right, font=\footnotesize] at (legend)  {#2};
}

% Draw subgraph (position, title)
\newenvironment{SubGraph}[2]{%
  \begin{pgfinterruptboundingbox} 
  \begin{scope}[shift={(#1)}]  
  \providecommand{\tmpSubGraphArgTwo}{#2}%
}{%
  \ifthenelse{\equal{\tmpSubGraphArgTwo}{}}
  {
    \coordinate (tmpNE) at (current bounding box.north east); 
    \coordinate (tmpSW) at (current bounding box.south west);   
  }{
    \coordinate (tmpNorth) at ($(current bounding box.north) + (0,0.4)$); 
    \coordinate (tmpNE) at ($(current bounding box.north east) + (0.2,0.2)$); 
    \coordinate (tmpSW) at ($(current bounding box.south west) - (0.2,0.2)$);   
    \draw[dashed] (tmpNE) rectangle (tmpSW);
    \node at (tmpNorth) {\tmpSubGraphArgTwo};
  }
  \end{scope}
  \end{pgfinterruptboundingbox}  
  \useasboundingbox (tmpNE) rectangle (tmpSW); 
}


% Draw graph figure (label, title)
\newenvironment{Graph}[2]{%
  \begin{figure}[ht!]%
  \providecommand{\tmpArgFirst}{#1}%
  \providecommand{\tmpArgSecond}{#2}%
  \begin{center}%
  \begin{tikzpicture}%
    \tikzstyle{every node}=[font=\footnotesize]%
    \tikzstyle{vertex}=[draw=black, inner sep=8pt]%
    \tikzstyle{edge}=[line width=0.7pt]%
}{%
  \end{tikzpicture}
  \end{center}
  \vspace*{-4mm}
  \caption{\tmpArgSecond}
  \label{\tmpArgFirst}
  \end{figure}
}
