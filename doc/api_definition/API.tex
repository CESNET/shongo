\documentclass[a4paper]{report}
\usepackage{geometry}
\geometry{paper=a4paper}

\usepackage[T1]{fontenc}
\usepackage[utf8]{inputenx}
\usepackage{palatino}
\usepackage{mathpazo}
\usepackage{microtype}
\renewcommand*\ttdefault{txtt}
\usepackage{eurosym}

\usepackage[czech,english]{babel}

\usepackage[pdftex,breaklinks=true,pdfborder={0 0 0}]{hyperref}
\usepackage{tocbibind} 
\usepackage{url}
\usepackage{paralist}
\usepackage{graphicx}
\usepackage{xtab}
\usepackage{booktabs}
\usepackage{calc}
\usepackage{ifthen}
\usepackage{xspace}

% this is to protect verbatim
\usepackage{cprotect}

\newcommand{\APIcmd}[1]{\item \texttt{#1}}

\newenvironment{APIdef}{\begin{itemize}}{\end{itemize}}

%{{{
\makeatletter
% we need to take only part of the \usecounter definition for DCcounter - no
% resets are wanted after end of the environment
\newcounter{UCcounter}
\setcounter{UCcounter}{0}
\newenvironment{UseCases}%
	{\begin{list}{UC-\arabic{UCcounter}}{\@nmbrlisttrue\def\@listctr{UCcounter}}}%
	{\end{list}}
\newcommand{\UClabel}[1]{\label{UC:#1}}
\newcommand{\UCref}[1]{UC-\ref{UC:#1}}

\newcommand{\UseCase}[2]{\item\UClabel{#2} \textbf{#1}\\}

\makeatother
%}}}

\usepackage{xcolor}
\newcommand{\TODO}[1]{%
\def\empty{}%
\def\prvniparametr{#1}%
\ifx\prvniparametr\empty%
\begingroup\tt\textcolor{red}{\noindent\textbf{TODO}}\endgroup
\else%
\begingroup\tt\textcolor{red}{\noindent\textbf{TODO:}\ #1}\endgroup
\fi%
} 


\begin{document}

\title{API for Shongo}
\author{Petr Holub, Jan Růžička, Miloš Liška, Martin Šrom, Ondřej Pavelka, Ondřej Bouda}
\date{\copyright~CESNET~z.\,s.\,p.\,o.\\2012}

\maketitle

\tableofcontents

\chapter{Use Cases}

\section{Reservations}

\begin{UseCases}

\UseCase{One Time reservation}{rsv:reservation:one}

Common type of reservation, where a user requests certain resources for limited
time duration. Unlimited reservations are not assumed by this scenario (see
\UCref{rsv:reservation:permanent}).

Start time of a reservation may be any time in the future or \emph{now}, which
is also called \emph{ad hoc} reservation.

Reserved resources may be given explicitly (exactly what amount of resourceson
which server) or implicitly.  The implicit resources are scheduled by Shongo
scheduler based on user input, e.g., 

\begin{compactitem}

\item amount of central resources (such as H.323 MCU ports or Connect licenses)
based on specified number of (H.323/SIP or web-browser) participants,

\item user may request a general endpoint and Shongo should try to find the
closest matching endpoint available to the user (e.g., user requests a H.323
endpoint for a conference since she has no personal endpoint, and she is
assigned a room-based H.323 endpoint provided the room is available),

\item any interconnecting elements (e.g., gateways) to interconnect the
endpoints specified by the user; if only part of the endpoints can be
interconnected, the user should be notified what parts can be interconnected
and what parts are disconnected.

\end{compactitem}

Each reservation has to be given a unique identifier that is further used for
any references to it.


\UseCase{Periodic reservation}{rsv:reservation:periodic}

\UCref{rsv:reservation:one} extended with periodicity. Expressiveness of the
periodicity language should be equivalent to cron plus explicit lists.

\UseCase{Permanent reservation}{rsv:reservation:permanent}

This is specific type of reservation that can be only made by an owner of the
resource as it permanently removes the reserved capacity from the dynamic
Shongo scheduling.

\UseCase{List all the reservations}{srv:list}

Some querying/filtering language needs to be supported to limit list to

\begin{UseCases}

\item specific room types (H.323, SIP, Connect, etc.),

\item specific reservation owner(s),

\item specific users (may be humans as well as service-users, such as rooms
with equipment) involved in the room as participants.

\end{UseCases}

\UseCase{Modification of a reservation}{rsv:modify}

\UseCase{Release/canceling of a reservation}{rsv:release}

\UseCase{Migration of a reservation}{rsv:migration}

\UseCase{Notify participants of a reservation}{rsv:notify-rsv}

All the (explicitly named) participants should receive an email that they are invited to participate in the room.

\UseCase{Notification of a user in case of reservation
change}{rsv:notify-change}

The notifications should be sent via email, but if user registers a cell phone
number, sending an SMS might also be very comfortable option.

\UseCase{Reservations of rooms, public or semi-private endpoints,
etc.}{rsv:service-users}

Each reservation may include service-users (beyond human users with private
endpoints---H.323/SIP/web), which represent entities such as rooms,
non-personal endpoints, etc., that can be scheduled in a similar way to central
resources.

\end{UseCases}


\section{Operations}

\begin{UseCases}

\UseCase{Live migration of a virtual room}{ops:migration}

This use case is intended for migration due to planned server maintenance or
unplanned server outage.  Ideally, all the room settings and content should be
transferred to the target room---but some content may be lost in case of
unplanned server failure (namely content migration).

Being able to transfer room settings to another server in case of unplanned
failure also requires that the settings needs to be stored in the Shongo
middleware.

Clients should be automatically redirected to the new server, if technology
permits, or at least notified of the migration (email, SMS---see
\UCref{rsv:notify-change}).

Some functionality will be common~\UCref{rsv:migration}.

\end{UseCases}

\subsection{Room Management}

\begin{UseCases}

\UseCase{List users}{ops:room:users-list}

Each user should be given a unique identifier in the output list that can be
used for further querying. It should also provide means to identify the same
user (e.g., if the user disconnects--reconnects, it should contain a part that
is common and that denotes the specific user and a part that is specific for
the session, so that if the user is connected twice (one session is in timeout
state and the other session has just been established), we can differentiate
between the two sessions).

\UseCase{Print detailed info about a user in a room}{ops:room:user-info}

Print all the statistics we can get about a user participating in the room. It
should contain technology agnostic part (e.g., when the user joined) and
technology specific part (e.g., H.323 statistics).

\UseCase{Set room layout}{ops:room:layout}

\UseCase{Disconnect a user}{ops:room:user-disconnect}

\UseCase{Mute a user}{ops:room:user-mute}

\UseCase{Set microphone audio level for a user}{ops:room:user-miclevel}

\UseCase{Set playback audio level for a user}{ops:room:user-playlevel}

\UseCase{Disable video of a user}{ops:room:user-video-off}

\UseCase{Set layout specific for a user}{ops:room:user-layout}

\end{UseCases}


\section{Monitoring \& Management}



\chapter{User Interface API Specification}

\chapter{Connector API Specification}

\section{Data Types}

\begin{APIdef}

\APIcmd{Date}

\APIcmd{Reservation}

\end{APIdef}


\section{Reservation API}

\begin{APIdef}

\APIcmd{Reservation makeReservation(Date date, int licenseCount)}

\end{APIdef}




\section{User Management API}

\begin{itemize}

\item

\end{itemize}


\section{Monitoring API}

\begin{itemize}

\item 

\end{itemize}


\section{Application Specific API}

\begin{itemize}

\item

\end{itemize}


\chapter{Inter-Controller API Specification}



\end{document}

