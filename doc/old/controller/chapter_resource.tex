\CodeStyle{} 
\CodeStyleAppendImplementation
\CodeStyleAppendEnum

\chapter{Resource Database}

Resource is physical or logical entity that can be scheduled by a scheduler. 
Resources are persistently stored in a relational database. Scheduler has 
access to that database and should keep the resources in memory in an 
efficient form for scheduling. Some of resources are created by Shongo users 
(e.g., videoconference devices or physical rooms), some are created by Shongo 
scheduler (e.g., virtual rooms or identifiers) and some are created due to a 
reservation request (e.g., unmanaged devices or resources from foreign 
domains). Resource database is composed of:
\begin{compactenum}
\item List of all known resources.
\item Device Topology that contains all device resources and represents 
videoconference reachability between them.
\end{compactenum}
Each resource can be one of the following types:
\begin{compactenum}
\item \textbf{Device} -- represents a videconference harware or software 
  element.
\item \textbf{Device Component} -- represents a resource that is a part of a 
  device resource. Can be one of the following types:
  \begin{compactenum}
    \item \textbf{Identifier}
    \item \textbf{Virtual Room}
  \end{compactenum}
\item \textbf{Physical Room} -- represent a physical room (e.g., classroom or 
  boardroom).
\end{compactenum}
Each resource can be dependent on another resource (e.g., can be located 
inside it or can be part of it), for instance a H.323 terminal can be located 
inside a physical room or a H.323 identifier is assigned to a H.323 device. 
When scheduler allocates resource, it must also allocate a resource to which 
the requested resource is dependent (recursively).

\section{Device Topology}

Device Topology is representation of videoconferencing devices from various 
technologies and its' mutual reachability in a single domain. Each domain 
controller in Shongo will keep in memory the local topology of all devices 
which the controller manages and which are part of controller reservations 
(unmanaged or foreign devices but only as single nodes without theirs 
topology).
Controller will not know topologies from foreign domains. Every time when the 
controller needs some information about foreign devices it send request to 
foreign domain controller.

Device Topology is composed of nodes and edges. Nodes represents 
videoconference devices and edges theirs reachability in a specific technology 
(e.g., ability to communicate in H.323). The reachability is determined by 
devices' capabilities and by list of global rules which defines the additional 
constraints.

\subsection{Node}

Each node in Device Topology represents one videoconference device. Device is 
able to communicate with one or more \textbf{technologies} (H.323, SIP, Adobe 
Connect, etc.).
\\
Device can have one or more \textbf{capabilities}. Each capability can specify 
one or more device \textbf{technologies} to which it is provided. By default a capability is provided to all device technologies. Capabilities 
can be from the following types:

\begin{itemize}
\newcommand{\RequireCapability}[1]{(requires #1)}
\newcommand{\ExtendCapability}[1]{(extends #1)}
  
\item \textbf{Receive}/\textbf{Send} \\
  Capability allows the device for receiving/sending videoconference data in
  input/output \textbf{format} that can specify:
  \begin{compactitem}
    \item \textbf{audio} in a specific \textbf{audio codecs} and 
      \textbf{bitrates}.
    \item \textbf{video} in a specific \textbf{video codecs} and 
      \textbf{resolutions}.
    \item \textbf{content} in a specific \textbf{content formats}.
  \end{compactitem}
  By default (if format is not specified) the device can receive/send all 
  streams (audio, video and content) in all codecs and resolutions.
  If format is specified the device can receive/send only
  given streams (audio, video and/or content) in specified 
  codecs and resolutions.
  
\item \textbf{Terminal} %
  \RequireCapability{\textbf{Send} and \textbf{Receive}} \\
  Capability tells that the device is able to participate in a 
  videoconference. Terminal devices must have \textbf{Receive} and 
  \textbf{Send} capability. The input/output \textbf{format} can be 
  specified once in the \textbf{Terminal} capability and it will be 
  copied/merged to both \textbf{Send} and \textbf{Received} capability in the 
  same device.
  
\item \textbf{Standalone Terminal} \ExtendCapability{\textbf{Terminal}} \\
  Capability tells that the device is a terminal which is able to communicate 
  to other standalone terminal directly without \textbf{Mix} or 
  \textbf{Virtual Rooms} device (e.g., H.323 and SIP terminals are generally 
  standalone terminal devices but Adobe Connect client is not standalone 
  terminal because it always needs the Adobe Connect server to start a 
  videoconference).
  
  \TODO{Technology can have specified that it can have only not-standalone devices,
  and persons can't select a device by which they will connect (e.g., Adobe Connect). 
  Or do we allow the person to select by which identity he will connect?}
  \\ \TODO{Each technology should have specified which types of devices it allows.}
  
\item \textbf{Mix} \RequireCapability{\textbf{Receive} and \textbf{Send}} \\
  Capability allows for mixing videoconference data from multiple 
  \textbf{Terminals} and send them back the combined result.
  
\item \textbf{Virtual Rooms} \ExtendCapability{\textbf{Mix}} \\
  Capability allows the device to host multiple virtual 
  rooms. \textbf{Terminal} devices will connect to these virtual rooms. 
  \textbf{Mixing} is applied per virtual room.

\item \textbf{Signaling Client} \\
  Capability makes the device to have set an address of a signaling server 
  which is used for initiating and/or routing the call. The device can
  also have set a list of aliases which identifies the device inside
  the signaling server zone and the server is able to translate the alias
  to URI or IP address.

\item \textbf{Signaling Server} \\
  Capability tells that the device is signaling server that can be used
  by other devices for initiating and/or routing the calls.

\item \textbf{Translate} \RequireCapability{\textbf{Receive} and \textbf{Send} 
  and at least 2 \textbf{technologies}} \\
  Capability allows for translation from one input format to another output 
  format where the first format is from one specified technology and the 
  second is from another specified technology. By default all translations 
  from input to output formats are allowed. The list of \textbf{rules} can be 
  used to restrict the allowed translations (e.g., for specific codecs, 
  resolutions, etc.).

\item \textbf{Stream} \RequireCapability{\textbf{Receive}} \\
  Capability tells that the device is able to perform streaming from an input 
  format.

\item \textbf{Record} \RequireCapability{\textbf{Receive}} \\
  Capability tells that the device is able to perform recording from an input 
  format.
\end{itemize}

Each device can have one or more described capabilities. If device supports 
for instance receiving/sending in multiple technologies, it can be implemented 
by multiple capability of the same type (one for each technology) or by single 
capability that lists all the technologies (provided that all technologies has 
same \textbf{format} settings).

\paragraph{Examples of abstract devices (each with technology dependent 
           examples):}

\begin{itemize}

\item \textbf{Terminal device} \\
  Terminal device is hardware or software client that is used by an user to 
  connect to a videoconference. The device allow an user to participate only 
  in one videoconference at the time. It has \textbf{Receive}, \textbf{Send} 
  and \textbf{Terminal} capability. It can have also the 
  \textbf{Standalone Terminal} capability which allows the terminal devices to  
  establish also 2-point videoconferences. 
  
\begin{EntityExample}{DeviceResource}{terminal1}%
      {Example of terminal for H.323 and/or SIP}
technologies: [H323, SIP],
ipAddress: 147.251.99.1,
capabilities: [
  StandaloneTerminalCapability,
  ReceiveCapability,
  SendCapability, 
]
\end{EntityExample}

\begin{EntityExample}{DeviceResource}{terminal2}%
      {Example of terminal (client) for Adobe Connect}
technologies: [AdobeConnect],
identity: srom@cesnet.cz,
capabilities: [
  TerminalCapability,
  ReceiveCapability,
  SendCapability
]
\end{EntityExample}

\begin{EntityExample}{DeviceResource}{terminal3}{Example of H323 terminal}
// It can process arbitrary audio, but video only in H.264. The video
// can be sent only in CIF - 720p resolution. It cannot process content.
technologies: [H323],
ipAddress: 147.251.99.3,
capabilities: [
  TerminalCapability {
    format: {
      audio: *,
      video: {codecs: [H264]}
    }
  },
  ReceiveCapability,
  SendCapability {
    format: {
      video: {resolutions: [CIF..720p]}
    }
  }
]
\end{EntityExample}

\item \textbf{Terminal device with virtual room} \\
  Terminal device with a single embedded virtual room is an extension of 
  terminal device that allows user to host a single multipoint 
  videoconference. The  device has \textbf{Receive}, \textbf{Send} and
  \textbf{Terminal} capability. It also has the \textbf{Mix} capability
  which provides a single virtual room in which the terminal device itself is 
  always participating.
  
\begin{EntityExample}{}{terminal4}%
      {Example of terminal with one virtual room for H.323 and/or SIP}
technologies: [H323, SIP],
ipAddress: 147.251.99.4,
capabilities: [
  TerminalCapability,
  ReceiveCapability,
  SendCapability
  MixCapability
]
\end{EntityExample}

\item \textbf{Multipoint device} \\
  Multipoint device is a special type of device that hosts one or more virtual 
  rooms and an arbitrary device can connect to a hosted virtual room and take 
  a videoconference there. The difference between multipoint and the previous 
  device is that the multipoint device can host more than one videoconference 
  and the multipoint device isn't part of these videoconferences, it only 
  manages them. The multipoint device has \textbf{Receive}, \textbf{Send}
  and \textbf{Virtual Rooms} capabilities.

\begin{EntityExample}{}{mcu1}{Example of H.323 and/or SIP multipoint device}
technologies: [H323, SIP],
ipAddress: 147.251.99.100,
capabilities: [
  ReceiveCapability,
  SendCapability,
  VirtualRoomsCapability
]
\end{EntityExample}

\begin{EntityExample}{}{server1}{Example of Adobe Connect server}
technologies: [AdobeConnect],
uri: obelix.cesnet.cz,
capabilities: [
  ReceiveCapability,
  SendCapability,
  VirtualRoomsCapability
]
\end{EntityExample}

\item \textbf{Signaling server device} \\
  Signaling server device is a special type of device that has only 
  \textbf{Signaling Server} capability. All of previous device examples can 
  also have \textbf{Signaling Client} capability which for instance allows 
  other devices to address them by it's alias.
    
\begin{EntityExample}{}{gatekeeper}{Example of gatekeeper for H.323}
technologies: [H323],
ipAddress: 147.251.99.101,
capabilities: [SignalingServerCapability]
\end{EntityExample}

\begin{EntityExample}{}{proxy}{Example of proxy for SIP}
technologies: [SIP],
ipAddress: 147.251.99.102,
capabilities: [SignalingServerCapability]
\end{EntityExample}

\begin{EntityExample}{}{mcu2}%
      {Example of MCU that is registered to H.323 and SIP servers}
technologies: [H323, SIP],
capabilities: [
  ReceiveCapability,
  SendCapability,
  VirtualRoomsCapability {rooms: [
    VirtualRoomResource {code: room1, aliases: [
      {technology: H323, type: E164, value: 9500000002},
      {technology: SIP, type: URI, value: sip:room1@cesnet.cz}
    ]}, 
    VirtualRoomResource {code: room2, aliases: [
      {technology: H323, type: E164, value: 9500000003}
      // room2 is not registered to SIP Proxy
    ]}
  ]},
  SignalingClientCapability {technologies: [H323], server: gatekeeper},  
  SignalingClientCapability {technologies: [SIP], server: proxy}
]
\end{EntityExample}

\begin{EntityExample}{}{terminal5}%
      {Example of terminal that is registered to H.323 and SIP servers}
technologies: [H323, SIP],
ipAddress: 147.251.99.5,
aliases: [
  {technology: H323, type: E164, value: 9500000001},
  {technology: SIP, type: URI, value: sip:srom@cesnet.cz}
],
capabilities: [
  TerminalCapability,
  ReceiveCapability,
  SendCapability,
  SignalingClientCapability {technologies: [H323], server: gatekeeper},
  SignalingClientCapability {technologies: [SIP], server: proxy}
]
\end{EntityExample}

\end{itemize}
 

\subsection{Edge}

The oriented edge is a link between two nodes and it tells that the second 
node (the node with incoming arrow) is reachable from the first node. A 
bidirectional edge tells that either node is reachable to the other node. 
Each edge is bound to a specific \textbf{technology} (e.g., H.323 or SIP) and 
to a specific \textbf{type} of connection in the technology (e.g., in H.323 
may be endpoints reachable by IP address or by H.323 alias, if both is 
available two edges must be present). Each edge also specifies the formats of data that are available.

\begin{EntityExample}{}{proxy}{}
technologies: [SIP],
capabilities: [SignalingServerCapability]
\end{EntityExample}

\begin{EntityExample}{}{terminal1}{}
technologies: [H323, SIP],
ipAddress: 147.251.99.1,
aliases: [{technology: SIP, type: URI, value: sip:terminal1}],
capabilities: [
  StandaloneTerminalCapability,
  ReceiveCapability, SendCapability,
  SignalingClientCapability {technologies: [SIP], server: proxy}
]
\end{EntityExample}

\begin{EntityExample}{}{terminal2}{}
technologies: [H323, SIP],
ipAddress: 147.251.99.2,
aliases: [{technology: SIP, type: URI, value: sip:terminal2}],
capabilities: [
  StandaloneTerminalCapability,
  ReceiveCapability, SendCapability,
  SignalingClientCapability {technologies: [SIP], server: proxy}
]
\end{EntityExample}

\begin{EntityExample}{}{edge1}{terminal1 <-> terminal2}
technology: H323,
type: IPAddress,
format: *
\end{EntityExample}

\begin{EntityExample}{}{edge2}{terminal1 <-> terminal2}
technology: SIP,
type: Alias,
format: *
\end{EntityExample} 
  
  
\subsection{Rule}
Rules in Device Topology are at global level and they are used for modifying 
the default reachability (determined by nodes capability settings). Each rule 
must specify a target \textbf{technology}, a \textbf{type} of connection in 
the technology (\verb|IPAddress| or \verb|Alias|), 
nodes (or groups of nodes) 
to which the rule is applied and modifier (\verb|Enable| or \verb|Disable|) 
which tells whether the reachability will be enabled or disabled. The nodes 
can be specified by pattern for identifier, IP address or alias. If more than 
one group of nodes is specified then the rule is applied only for edges 
between the groups and not inside the groups.

\begin{EntityExample}{}{rule1}%
      {Disable reachability by IP address for all H.323 devices}
technologies: [H323],
types: [IPAddress],
nodes: *,
goals: Disable
\end{EntityExample}

\begin{EntityExample}{}{rule2}%
      {Enable reachability by IP address for all H.323 devices at FI MUNI}
technologies: [H323],
types: [IPAddress],
nodes: cz.muni.fi.*,
goal: Enable
\end{EntityExample}

\begin{EntityExample}{}{rule3}{}
// Enable reachability by alias of SIP and H.323 devices between two networks.
// Devices from each network are referenced by the network signaling server.
technologies: [H323, SIP],
types: [Alias],
nodes: [signalingServer1.*, signalingServer2.*],
goal: Enable
\end{EntityExample}

\begin{EntityExample}{}{rule3}{Enable all correct reachability}
technologies: *,
types: *,
nodes: *,
goal: Enable
\end{EntityExample}

%\section{Operations}

%Device Topology implementation should allow dynamic modification when for 
%instance some device parameters are changed to not force the whole topology 
%reconstruction.      
      
%\subsection*{Construction}

%Construction is performed when the controller is started. Steps to construct 
%a topology:

%\begin{compactenum}
%\item Get the list of all devices in the topology.
%\item For each device: 
%  \begin{compactitem}
%  \item \textbf{Add new device}.
%  \end{compactitem}
%\end{compactenum}     

%\subsection*{Add new device}     

%Steps to add new device to the topology:
%\begin{compactenum}
%\item Get device capabilities.
%\item Add a new node to the topology.
%\item Find all other devices that the device is able to take a %
%  videoconference with (based on capabilities):
%  \begin{compactitem}
%  \item Add edge(s) with proper parameters.
%  \end{compactitem}
%\end{compactenum}  

%\subsection*{Update existing device}

%Steps to update an existing device in a topology for instance when the %capabilities for the device are changed:
%\begin{compactenum}
%\item \textbf{Remove existing device}.
%\item \textbf{Add new device}.
%\end{compactenum} 
     
%\subsection*{Remove existing device}     

%Steps to remove existing device from a topology:
%\begin{compactenum}
%\item Remove all edges that reference the device.
%\item Remove the node from the topology.
%\end{compactenum}  

\section{Implementation}

The resource database is implemented as follows:

\CodeInput{code_resource.txt}

\section{Operations}

Operations that can be performed on resource database:

\begin{description}
\item[Construct]
Resource database is constructed when the controller (and thus the scheduler) 
is started. The scheduler loads all resources from relational database and 
performs \textbf{Add new resource} for each loaded resource.

\item[Add new resource]
The scheduler adds the resource to the list of all resources. If the resource 
is device type then the resource is also added to the Device Topology.

\item[Remove existing resource]
The scheduler removes the resource from the list of all resources. If the 
resource is device type then the resource is also removed from the Device 
Topology.
\end{description} 
