\chapter{Technology}

This chapter describes widely-used videoconference technologies. Types of 
network elements and network examples are listed for each technology.

\section{H.323}

The H.323 network is composed of terminals, multipoint control units, gateways 
and gatekeepers. Terminals, MCUs and gateways are referred to as endpoints. At 
least two terminals are needed to start a videoconference call.

\subsection{Elements}

\begin{itemize}

\item \textbf{Terminal} \\
Terminal represents a hardware device or a software client that enables a 
single person or a group of persons situated in front of the device to connect 
to another H.323 endpoint and take a videoconference call.

Some terminals may be equipped with MCU functionality for one virtual room and thus connect multiple other endpoints to a single videoconference. Terminal that provides described functionality always participate in the provided virtual room.

There are special terminal devices that only receives videoconference 
channels and are used to stream or record them. They are referred to as 
streaming or recording servers.

\item \textbf{Multipoint Control Unit} (MCU) \\
MCU represents a hardware (or software) device that can host multiple virtual rooms.  In each virtual room can more than two endpoints participate in one videoconference session simultaneously.
\TODO{How can MCU restrict number of participants? DSP/license (sometimes resolution based)...}

\item \textbf{Gateway} \\
Gateway is device that enable communication between H.323 network and other 
networks (e.g., PSTN).

\item \textbf{Gatekeeper} \\
Gatekeeper is a functionality device that provides special services to the other 
elements in the network: endpoint registration, address resolution, admission 
control and user authentication. It can limit the number of connections that 
are established through it.

\end{itemize}

Each H.323 element is running on a specific IP address that can be used to 
connect to the element. An element can have assigned alias, \textbf{H.323 
phone number} (defined by E.164), \textbf{H.323 identifier} (string, e.g., 
email) or \textbf{H.323 URI} (e.g., \verb|name@domain|), that can be used to connect to it. Alias can be assigned only to \textbf{terminal} or \textbf{virtual room} in 
MCU. When a terminal has assigned alias it must register to a 
\textbf{gatekeeper} that will be able to translate the alias to the terminal 
IP address. Similarly the MCU must register aliases for it's virtual rooms to 
a gatekeeper. When an element want to connect to another terminal or virtual 
room by alias, it must use the same gatekeeper that has the proper registry 
of aliases or the gatekeeper must be configured to route the request to a 
proper other gatekeeper.

When a gatekeeper is used in videoconference, it can work in two modes:
\begin{itemize}
\item Direct Endpoint Model -- The gatekeeper is used only for initialization 
(address resolution, admission control, etc.) and the videoconference call 
itself is managed by it's participants (endpoints). The gatekeeper doesn't 
have full control over the call signaling. The target IP address is 
communicated to the caller.
\item Gatekeeper Routed Model -- The gatekeeper is used throughout 
videoconference call, the call signaling flow through the gatekeeper and it 
has the full control over it. The target IP address is not communicated to 
the caller.
\end{itemize}

Each terminal and MCU element can have set one main gatekeeper and one backup 
gatekeeper. \TODO{Or more backups? Does backup have different registry of 
aliases?} If gatekeeper is set then each terminal and virtual room in MCU 
can have an alias set.

\subsection{Constraints}

\begin{itemize}
\item Communication consists of:
\begin{compactenum}
\item Audio
\item Audio + Video
\item Audio + Content
\item Audio + Video + Content
\end{compactenum}

\item Bidirectional accessibility. When an element $A$ is accessible to an 
element $B$ then also $B$ is accessible to $A$.
\end{itemize}

\subsection{Network Examples}

This section shows some examples of H.323 networks via graphs where nodes represents H.323 elements and edges represents available channels or reachability that are worth to mention. Edge does not mean that it is present in a specific videoconference. For instance when MCU is used in videoconference then "Connect by ..." edges are used only between each participating terminal and MCU and not between terminals mutually. The graphs  show all edges (channels and reachability) that can be used by any videoconference.

% H.323 Graph Declarations
\tikzstyle{hxxxData}=[color=red]
\tikzstyle{hxxxDataDashed}=[color=red, dashed]
\tikzstyle{hxxxControl}=[color=orange]
\newcommand{\HXXXLegend}[1]{
  \begin{GraphLegend}{#1}
    \GraphLegendItem{hxxxData}{Call data flow}
    \GraphLegendItem{hxxxControl}{Call control flow}
    \GraphLegendItem{hxxxAlias}{Connect by H.323 alias}
    \GraphLegendItem{hxxxIp}{Connect by IP address}
  \end{GraphLegend}
}

\subsubsection{Connection through direct IP addresses}

Terminals and virtual rooms in MCU don't have assigned any H.323 aliases thus 
the connecting is done through IP addresses. This topology allows for 2-point 
videoconferences (e.g., \emph{terminal1 -- terminal2}) and multipoint 
videoconferences through the MCU (e.g., \emph{terminal1 -- terminal2 -- 
terminal3 -- mcu}). When a terminal connects to the MCU it must select the 
proper virtual room, because the IP address determines only the MCU not a 
specific virtual room. See fig. \ref{graph:h323:direct}.

\begin{Graph}{graph:h323:direct}{Connection through direct IP addresses}
  \Vertex{0, 0}{t1}{terminal1}
  \Vertex{4, 0}{t2}{terminal2}
  \Vertex{4,-4}{t3}{terminal3}
  \Vertex{0,-4}{mcu}{mcu}
  
  \EdgeAllToAll{<->}{hxxxIp, hxxxData, hxxxControl}{t1, t2, t3, mcu}
  
  \HXXXLegend{6, -1}
\end{Graph}

\subsubsection{Connection through a gatekeeper (Direct Endpoint Model)}

Terminals and MCUs are configured to forbid direct connections through IP 
addresses. Terminals and MCU virtual rooms have assigned H.323 aliases and the 
devices register to the gatekeeper that is able to resolve these aliases to 
direct IP addresses. When a terminal want to start videoconference he must 
request the gatekeeper. The request must contain H.323 alias of another 
terminal or virtual room in MCU to which the terminal wants to connect. The 
gatekeeper resolves an IP address from the alias and return it to the caller 
which then is able to start videoconference data flow to the requested target. 
The gatekeeper can perform admission control and forbid the connection. When a 
videoconference ends the caller informs the gatekeeper again. The gatekeeper 
holds the list of active connections and thus can limit the maximum number of 
connections at the same time. See fig. \ref{graph:h323:gatekeeper}.

\begin{Graph}{graph:h323:gatekeeper}{Connection through gatekeeper}
  \Vertex{ 0, 0}{t1}{terminal1}
  \Vertex{ 4, 0}{t2}{terminal2}
  \Vertex{ 4,-4}{t3}{terminal3}
  \Vertex{ 0,-4}{mcu}{mcu}
  \Vertex{-2,-2}{gk}{gatekeeper}
  
  \EdgeAllToAll{<->}{hxxxData, hxxxControl}{t1, t2, t3, mcu}
  \EdgeOneToAll{<->}{hxxxAlias}{gk}{t1, t2, t3, mcu}
  
  \HXXXLegend{6, -1}
\end{Graph}

\subsubsection{Connection through a gatekeeper (Gatekeeper Routed Model)}

Terminals and MCUs are configured to forbid direct connections through IP 
addresses. Terminals and MCU virtual rooms have assigned H.323 aliases and the 
devices register to the gatekeeper that is able to resolve these aliases to 
direct IP addresses. When a terminal want to start videoconference he must 
request the gatekeeper. The request must contain H.323 alias of another 
terminal or virtual room in MCU to which the terminal wants to connect. The 
gatekeeper resolves an IP address from the alias and establish connection to 
the target endpoint. The call is then mediated by the gatekeeper throughout 
the videoconference. The gatekeeper can also perform admission control and 
limit the number of active connections. In special cases even the data flow 
can be routed through the gatekeeper (e.g., when connecting endpoints behind 
the NAT). See fig. \ref{graph:h323:gatekeeperRouted}.

\begin{Graph}{graph:h323:gatekeeperRouted}{Connection through gatekeeper}
  \Vertex{0, 0}{t1}{terminal1}
  \Vertex{4, 0}{t2}{terminal2}
  \Vertex{4,-4}{t3}{terminal3}
  \Vertex{0,-4}{mcu}{mcu}
  \Vertex{-2,-2}{gk}{gatekeeper}
  
  \EdgeAllToAll{<->}{hxxxData}{t1, t2, t3, mcu}  
  \EdgeOneToAll{<->}{hxxxAlias, hxxxControl}{gk}{t1, t2, t3, mcu}
  
  \HXXXLegend{6, -1}
\end{Graph}

\subsubsection{Multiple gatekeepers}

The network is composed from 3 endpoints (\emph{terminal1, terminal2, mcu}) 
managed by \emph{gatekeeper1} and 2 subdomains where each is composed of 2 
endpoints and one gatekeeper. Adjacent endpoints (endpoints in the same local 
network) can connect directly or through it's gatekeeper. Not adjacent 
endpoints (endpoints from different local networks) can connect only by it's 
gatekeepers. See fig. \ref{graph:h323:gatekeeperMultiple}.

\begin{Graph}{graph:h323:gatekeeperMultiple}{Multiple gatekeepers}
  \Vertex{ 0, 0}{gk1}{gatekeeper1}
  \Vertex{-2,-2}{gk2}{gatekeeper2}
  \Vertex{ 2,-2}{gk3}{gatekeeper3}
  \EdgeAllToAll{<->}{hxxxAlias, hxxxControl}{gk1, gk2, gk3}
  
  \Vertex{-2, 2}{t1}{terminal1}
  \Vertex{ 2, 2}{t2}{terminal2}
  \Vertex{ 0, 4}{mcu}{mcu}
  \EdgeAllToAll{<->}{hxxxIp, hxxxControl, hxxxData}{t1, t2, mcu}  
  \EdgeOneToAll{<->}{hxxxAlias, hxxxControl}{gk1}{t1, t2, mcu}  

  \Vertex{-5,-1}{t3}{terminal3}
  \Vertex{-5,-3}{t4}{terminal4}
  \Edge{<->}{hxxxIp, hxxxControl, hxxxData}{t3}{t4}
  \EdgeOneToAll{<->}{hxxxAlias, hxxxControl}{gk2}{t3, t4}  
     
  \Vertex{5,-1}{t5}{terminal5}
  \Vertex{5,-3}{t6}{terminal6}
  \Edge{<->}{hxxxIp, hxxxControl, hxxxData}{t5}{t6}
  \EdgeOneToAll{<->}{hxxxAlias, hxxxControl}{gk3}{t5, t6}    
  
  \Edge{<-}{hxxxDataDashed}{mcu}{$(mcu) + (0,1)$}
  \Edge{<-}{hxxxDataDashed}{t1}{$(t1) + (-1.5,0)$}
  \Edge{<-}{hxxxDataDashed}{t2}{$(t2) + (1.5,0)$}
  \Edge{<-}{hxxxDataDashed}{t3}{$(t3) + (0.3, 1)$}
  \Edge{<-}{hxxxDataDashed}{t4}{$(t4) + (0.8,-1)$}
  \Edge{<-}{hxxxDataDashed}{t5}{$(t5) + (-0.3, 1)$}
  \Edge{<-}{hxxxDataDashed}{t6}{$(t6) + (-0.8,-1)$}

  \HXXXLegend{4.5, 4.5}
\end{Graph}

\subsection{Suggested Topology}
In Topology should be present only endpoints (terminals, MCUs, gateways). The 
gatekeepers will be kept aside and will be used only to determine which 
devices are accessible by which devices and to limit the number of active 
connections that will flow through them (if maximum number of connections is 
specified for the gatekeeper). See fig. \ref{graph:h323:topology}.

\begin{Graph}{graph:h323:topology}{Suggested topology for H.323}  
  \Vertex{-2, 4}{gk1}{gatekeeper1}
  \Vertex{-2, 2}{t1}{terminal1}
  \Vertex{-2,-2}{t2}{terminal2}
  \Vertex{-4, 0}{mcu}{mcu}

  \Vertex{2, 4}{gk2}{gatekeeper2}  
  \Vertex{2, 2}{t3}{terminal3}
  \Vertex{2,-2}{t4}{terminal4}
  
  \node at (-2, 3) {\textit{H.323 zone 1}};
  \node at ( 2, 3) {\textit{H.323 zone 2}};
  \draw[dashed, color=gray](0,4.6) -- (0,-2.5);
    
  \EdgeAllToAll{<->}{hxxxAlias, hxxxIp}{t1, t2, mcu}  
  \EdgeAllToAll{<->}{hxxxAlias, hxxxIp}{t3, t4}  
  \EdgeOneToAll{<->}{hxxxAlias}{t3}{t1, t2, mcu}  
  \EdgeOneToAll{<->}{hxxxAlias}{t4}{t1, t2, mcu}  
  
  \begin{GraphLegend}{4, 0.5}
    \GraphLegendItem{hxxxAlias}{Accessible by H.323 alias}
    \GraphLegendItem{hxxxIp}{Accessible by IP in H.323}
  \end{GraphLegend}
\end{Graph}

\paragraph{Notes:}
\begin{itemize}
\item At the global level there should be a list of rules that defines which 
devices are accessible to which devices by IP address (similarly to firewall 
definition). By default all H.323 endpoints are accessible to all by IP 
address (even with different assigned gatekeepers).
\item At the global level there should be a list of rules that defines which 
devices are accessible to which devices by alias. Rules should primarily use 
gatekeepers for specifying accessibility (e.g., endpoints from the first 
gatekeeper are accessible to endpoints from the second gatekeeper). By default 
endpoints are accessible only inside the same gatekeeper. The accessibility 
between two gatekeepers must be explicitly defined by a rule.
\end{itemize}

\subsection{References}

\renewcommand{\bibsection}{}
\begin{thebibliography}{1}
\bibitem[1]{bib:h323:wiki}
H.323 on Wikipedia.
\\\url{https://en.wikipedia.org/wiki/H.323}

\bibitem[2]{bib:h323:architecture}
Basic Architecture of H.323.
\\\url{http://hive1.hive.packetizer.com/users/packetizer/papers/h323/h323_basics_handout.pdf}

\bibitem[3]{bib:h323:gatekeepers}
Understanding H.323 Gatekeepers.
\\\url{http://www.cisco.com/en/US/tech/tk1077/technologies_tech_note09186a00800c5e0d.shtml}

\bibitem[4]{bib:h323:seminar}
Seminář IP telefonie.
\\\url{http://www.cesnet.cz/akce/20051115/pr/voz02_h323.pdf}
\end{thebibliography}

\section{SIP}

The Session Initiation Protocol (SIP) is a signaling protocol for establishing 
calls and videoconferences over IP networks.

\subsection{Elements}

Type of elements in SIP network:

\begin{itemize}
\item \textbf{User Agent} \\
User agent is a hardware or software client that allows user to take a 
videoconference call. It can create and receive SIP messages and thereby
manage a SIP session.

Each user agent should have set precisely one registrar to which it registers 
and it will be available through it to other agents.

\item \textbf{Multipoint Control Unit} (MCU) \\
MCU represents the same hardware device as in H.323. It contains one or more 
\textbf{virtual rooms} and each virtual room can host one videoconference call 
at the time.

A single MCU device can support both SIP and H.323 technology which means that 
to a single virtual room on that MCU can connect SIP user agents and also 
H.323 terminals. Each virtual room can be configured to support both 
technologies or only one of these technologies.

\item \textbf{Proxy Server} \\
Proxy server is an intermediary entity whose main role is to route requests to 
another entity that is closer to the target. Proxy server can also perform 
admission control.

\item \textbf{Redirect Server} \\
Redirect server response to each client request by alternative set of URIs 
which should client contact instead of the original target.

\item \textbf{Registrar} \\
Registrar is an element that keeps register of user agents. Each user agent 
can register itself to a registrar for a specific URI. More user agents can be 
registered to a single URI. 

For scheduling purposes we combine the previous three servers (Proxy, Redirect 
and Registrar) to a single server referred to as SIP Proxy.

\item \textbf{Gateway} \\
Gateway is a device that enable communication between SIP network and other 
networks (e.g., PSTN).
\end{itemize}

Each user agent in a SIP network is identified by an uniform resource 
identifier (URI), for instance \verb|sip:username:password@host:port|. 
Multiple user agents can register to registrar with a single URI and when the 
URI is requested by a caller all devices will "ring" and only one can answer 
the call (e.g., one user with single URI can have multiple phones, hardware or 
software, by which he can answer an incoming call).

The URI can contain the target IP address. When an user agent use this type of 
URI to start a call, user agent's Proxy (combination of Registrar, Proxy 
and/or Redirect Server) is still used to initiate the call unlike in H.323 
where the connection can be done directly through the IP address without 
notifying a gatekeeper.

\subsection{Constraints}

\begin{itemize}

\item Communication consists of:
\begin{compactenum}
\item Audio
\item Audio + Video
\item Audio + Video + Content
\end{compactenum}

\item Bidirectional accessibility. When an element $A$ is accessible to an 
element $B$ then also $B$ is accessible to $A$.

\item Each user agent must have the the Registrar set.

\end{itemize}

\subsection{Network Examples}

SIP network is very similar to H.323 network and doesn't bring any new aspects 
or elements thus this section can be skipped.

\subsection{Suggested topology}

For SIP network we will use the same naming convention as for H.323 and 
endpoints will stand for user agents (terminals), MCUs and gateways. Proxy 
Server, Redirect Server and Registrar are grouped into a single element SIP 
Proxy.%
\begin{Graph}{graph:sip:topology}{Suggested topology for SIP}  
  \Vertex{-2, 4}{gk1}{proxy1}
  \Vertex{-2, 2}{a1}{userAgent1}
  \Vertex{-2,-2}{a2}{userAgent2}
  \Vertex{-4, 0}{mcu}{mcu}

  \Vertex{2, 4}{gk2}{proxy2}  
  \Vertex{2, 2}{a3}{userAgent3}
  \Vertex{2,-2}{a4}{userAgent4}
  
  \node at (-2, 3) {\textit{SIP zone 1}};
  \node at ( 2, 3) {\textit{SIP zone 2}};
  \draw[dashed, color=gray](0,4.6) -- (0,-2.5);
    
  \EdgeAllToAll{<->}{sipUri}{a1, a2, mcu, a3, a4}  
  
  \begin{GraphLegend}{4, 0.25}
    \GraphLegendItem{sipUri}{Accessible by SIP URI}
  \end{GraphLegend}
\end{Graph}
In Device Topology should be present only endpoints. SIP Proxy elements are 
kept aside and they are used only when the topology is constructed to 
determine which endpoints are accessible from which endpoints. See fig. 
\ref{graph:sip:topology}.

\paragraph{Notes:}
\begin{itemize}
\item Similarly like in H.323, at the global level there should be a list of 
rules that defines which devices are accessible to which devices by URI. By 
default endpoints are accessible only inside the same Proxy. The accessibility 
between two Proxies must be explicitly defined by a rule.
\end{itemize}

\subsection{References}

\renewcommand{\bibsection}{}
\begin{thebibliography}{1}

\bibitem[1]{bib:sip:architecture}
Session Initiation Protocol (SIP)
\\ \url{http://www.vide.net/cookbook/cookbook.en/list_page.php?
topic=3&url=sip.htm}

\bibitem[2]{bib:sip:wiki}
Session Initiation Protocol on Wikipedia.
\\ \url{https://en.wikipedia.org/wiki/Session_Initiation_Protocol}

\end{thebibliography}

\section{Adobe Connect}

Adobe Connect is a web conferencing software for web meetings, eLearning, and 
webinars. The software is running as a server on a specific URL. It contains 
several meetings (virtual rooms) which can be accessed by clients.

\subsection{Elements}

Type of elements in Adobe Connect network:

\begin{itemize}
\item \textbf{Client} \\
Client for Adobe Connect is a web browser with Adobe Flash Player installed or 
Adobe Connect Add-in. To be able to connect two clients, the Adobe Connect 
server with created meeting is needed as opposed to H.323/SIP  where two 
terminals/user agents are sufficient to start a videoconference.

Special type of client \textbf{Gateway} can exist. This client can translate 
Adobe Connect videoconference to other technology (e.g., SIP or H.323).

\item \textbf{Server} \\
Server is an installed Adobe Connect software that is made accessible on a 
specific URL. On the server one or multiple meetings (virtual rooms) can be 
created. Clients are then able to connect to a meeting on a server. Each 
virtual room can be accessed by the server URL followed by a meeting name 
(e.g., \verb|obelix.cesnet.cz/<MEETING_NAME>|).
\end{itemize}

Adobe Connect servers cannot be cascaded (connected together) as opposed to 
H.323/SIP MCUs which can be cascaded. Thus the Adobe Connect videoconference 
scheduling will be simpler task than scheduling for other described 
technologies.

Clients can be identified by shibboleth (e.g., eduID.cz). Managed client 
doesn't make sense, it would be controlling application for a web browser. 
Managed client only make sense for gateway client (e.g, client that translate 
Adobe Connect channels to SIP or H.323).

\subsection{Constraints}

\begin{itemize}
\item Communication consists of:
\begin{compactenum}
\item Audio
\item Audio + Video
\item Audio + Video + Content
\end{compactenum}
\end{itemize}

\subsection{Suggested topology}

In Topology should be present both clients and servers. Edges will be only 
between clients and servers and not between clients themselves. When two 
clients want to take a videoconference call they both must connect to the same 
Adobe Connect meeting on the same server.

\begin{Graph}{graph:connect:topology}{Suggested topology for Adobe Connect}  
  \Vertex{-2, 3}{s1}{server1}
  \Vertex{-3, 0}{c1}{client1}
  \Vertex{-1, 0}{c2}{client2}

  \Vertex{2, 3}{s2}{server2}  
  \Vertex{1, 0}{c3}{client3}
  \Vertex{3, 0}{c4}{client4}
      
  \EdgeOneToAll{<-}{connectUrl}{s1}{c1, c2, c3, c4}  
  \EdgeOneToAll{<-}{connectUrl}{s2}{c1, c2, c3, c4}  
  
  \begin{GraphLegend}{4.5, 1.75}
    \GraphLegendItem{connectUrl}{Accessible by Adobe Connect URL}
  \end{GraphLegend}
\end{Graph}

\subsection{References}

\renewcommand{\bibsection}{}
\begin{thebibliography}{1}

\bibitem[1]{bib:connect:architecture}
Adobe Connect Official Website.
\\ \url{http://www.adobe.com/products/adobeconnect.html}

\bibitem[2]{bib:connect:wiki}
Adobe Connect on Wikipedia.
\\ \url{http://en.wikipedia.org/wiki/Adobe_Connect}

\end{thebibliography}
